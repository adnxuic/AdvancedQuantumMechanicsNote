\chapter{Calculation of the Determinant and Inverse of the a Special Matrix}
\label{appendix:special-matrix}

Let the given $n\times n$ matrix be denoted by $A_n$.
\begin{equation}
A_n =
    \begin{bmatrix}
    2 & -1 & & \\
    -1 & 2 & -1 & \\
    & \ddots & \ddots & \ddots \\
    & & -1 & 2 & -1 \\
    & & & -1 & 2
    \end{bmatrix}_{n\times n}
\end{equation}

\section{Calculation of det(A)}

Let $D_n$ denote the determinant of an $n \times n$ matrix of the form described in the problem. The matrix $A$ given in the problem has dimensions $(N-1) \times (N-1)$. For the sake of recurrence derivation, we analyze the matrix $A_n$ of size $n \times n$:

\begin{equation}
A_n = \begin{bmatrix}
2 & -1 & 0 & \dots & 0 \\
-1 & 2 & -1 & \dots & 0 \\
0 & -1 & 2 & \dots & 0 \\
\vdots & \vdots & \vdots & \ddots & \vdots \\
0 & 0 & 0 & -1 & 2
\end{bmatrix}_{n \times n}
\end{equation}

\textbf{Step 1: Find a recurrence relation}

We compute the determinant using cofactor expansion along the first row. Let $D_n = \det(A_n)$.

\begin{equation}
D_n = 2 \cdot \det(M_{1,1}) - (-1) \cdot \det(M_{1,2})
\end{equation}

Here, $M_{i,j}$ represents the minor matrix obtained by removing row $i$ and column $j$.
\begin{enumerate}
    \item Removing row 1 and column 1 leaves a matrix of the same structure with size $(n-1) \times (n-1)$, so $\det(M_{1,1}) = D_{n-1}$.
    \item Removing row 1 and column 2 leaves a matrix where the first column contains only $-1$ in the top position and zeros elsewhere. Expanding along this new first column yields $-1$ multiplied by the determinant of the remaining $(n-2) \times (n-2)$ matrix (which again has the original structure). Thus, $\det(M_{1,2}) = -1 \cdot D_{n-2}$.
\end{enumerate}

Substituting these back into the expansion equation:

\begin{equation}
\begin{aligned}
D_n &= 2(D_{n-1}) - (-1)(-1 \cdot D_{n-2}) \\
D_n &= 2D_{n-1} - D_{n-2}
\end{aligned}
\end{equation}

\textbf{Step 2: Calculate base cases}

\begin{equation}
\begin{aligned}
\text{For } n=1: & \quad A_1 = [2] \implies D_1 = 2 \\
\text{For } n=2: & \quad A_2 = \begin{bmatrix} 2 & -1 \\ -1 & 2 \end{bmatrix} \implies D_2 = (2)(2) - (-1)(-1) = 3
\end{aligned}
\end{equation}

\textbf{Step 3: Solve the recurrence}

We observe the pattern $D_1 = 2$, $D_2 = 3$, $D_3 = 4$. We hypothesize that $D_n = n + 1$.
We prove this by induction.
\begin{enumerate}
    \item The hypothesis holds for base cases $n=1$ and $n=2$.
    \item Assume $D_k = k+1$ for all $k < n$.
    \item Apply the recurrence:
\end{enumerate}

\begin{equation}
\begin{aligned}
D_n &= 2 D_{n-1} - D_{n-2} \\
&= 2((n-1) + 1) - ((n-2) + 1) \\
&= 2n - (n-1) \\
&= 2n - n + 1 \\
&= n + 1
\end{aligned}
\end{equation}

\textbf{Step 4: Final Calculation}

The matrix $A$ has size $(N-1) \times (N-1)$. Therefore, we substitute $n = N - 1$ into the formula:

\begin{equation}
\det(A) = D_{N-1} = (N - 1) + 1 = N
\end{equation}

\section{Calculation of \texorpdfstring{$A^{-1}$}{A-1}}

Let $B = A^{-1}$. The elements of the inverse matrix are denoted as $b_{i,j}$. From the definition $AB = I$, for a fixed column $j$ of the inverse matrix, the elements $x_i = b_{i,j}$ satisfy the linear system:

\begin{equation}
A \bm{x} = \bm{e}_j
\end{equation}

where $\bm{e}_j$ is the standard basis vector. The indices $i$ and $j$ run from $1$ to $N-1$.

\textbf{Step 1: Set up the difference equations}

The structure of $A$ corresponds to a discrete second derivative. The equation for the $i$-th row is:

\begin{equation}
-x_{i-1} + 2x_i - x_{i+1} = \delta_{i,j}
\end{equation}

The boundary conditions correspond to the nodes outside the matrix:

\begin{equation}
x_0 = 0 \quad \text{and} \quad x_N = 0
\end{equation}

\textbf{Step 2: Solve the homogeneous regions}

For $i \neq j$, the equation is $-x_{i-1} + 2x_i - x_{i+1} = 0$, which implies $x_{i+1} - x_i = x_i - x_{i-1}$. The solution is linear in $i$.

\begin{enumerate}
    \item \textbf{Region 1 ($0 \le i \le j$):} Applying $x_0 = 0$, the solution is $x_i = c_1 i$.
    \item \textbf{Region 2 ($j \le i \le N$):} Applying $x_N = 0$, the solution is $x_i = c_2 (N - i)$.
\end{enumerate}

\textbf{Step 3: Match continuity at i = j}

The solution must be continuous at $i=j$:

\begin{equation}
c_1 j = c_2 (N - j) \implies c_2 = c_1 \frac{j}{N-j}
\end{equation}

\textbf{Step 4: Apply the jump condition at i = j}

Substitute the neighbors into the non-homogeneous equation at $i=j$:

\begin{equation}
-x_{j-1} + 2x_j - x_{j+1} = 1
\end{equation}

Using the linear forms derived above ($x_{j-1}$ from Region 1, $x_{j+1}$ from Region 2):

\begin{equation}
\begin{aligned}
-[c_1(j-1)] + 2[c_1 j] - [c_2(N - j - 1)] &= 1 \\
c_1(j+1) - c_2(N - j - 1) &= 1
\end{aligned}
\end{equation}

Substitute $c_2 = c_1 \frac{j}{N-j}$:

\begin{equation}
\begin{aligned}
c_1(j+1) - \left( c_1 \frac{j}{N-j} \right) (N - j - 1) &= 1 \\
c_1 \left[ (j+1) - \frac{j(N - j - 1)}{N-j} \right] &= 1
\end{aligned}
\end{equation}

Multiplying by $(N-j)$ to clear the denominator:

\begin{equation}
\begin{aligned}
c_1 \left[ (j+1)(N-j) - j(N - j - 1) \right] &= N - j \\
c_1 \left[ (jN - j^2 + N - j) - (jN - j^2 - j) \right] &= N - j \\
c_1 \left[ N \right] &= N - j
\end{aligned}
\end{equation}

Thus, we find the coefficients:

\begin{equation}
c_1 = \frac{N-j}{N}, \quad c_2 = \frac{j}{N}
\end{equation}

\textbf{Step 5: Final Matrix Elements}

Substituting the constants back into the piecewise expressions:
For $i \le j$: $x_i = \frac{N-j}{N} i$.
For $i \ge j$: $x_i = \frac{j}{N} (N-i)$.

The elements of the inverse matrix are given by:

\begin{equation}
(A^{-1})_{i,j} = \frac{1}{N} \min(i,j) (N - \max(i,j))
\end{equation}