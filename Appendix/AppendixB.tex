\chapter{Calculation of the Determinant and Inverse of the a Special Matrix}
\label{appendix:special-matrix}

Let the given $n\times n$ matrix be denoted by $A_n$.
\begin{equation}
A_n =
    \begin{bmatrix}
    2 & -1 & & \\
    -1 & 2 & -1 & \\
    & \ddots & \ddots & \ddots \\
    & & -1 & 2 & -1 \\
    & & & -1 & 2
    \end{bmatrix}_{n\times n}
\end{equation}

\section{Calculation of the Determinant}

We will find the determinant by establishing a recurrence relation. 
Let $D_n = \det(A_n)$ be the determinant of the $n \times n$ version of this matrix.


\textbf{Determinants for Small Sizes}

We compute the determinant for small values of $n$ to identify a pattern.
\begin{itemize}
    \item For $n=1$:
    \begin{equation}
        A_1 = \begin{bmatrix} 2 \end{bmatrix} \implies D_1 = \det(A_1) = 2
    \end{equation}
    \item For $n=2$:
    \begin{equation}
        A_2 = \begin{bmatrix} 2 & -1 \\ -1 & 2 \end{bmatrix} \implies D_2 = \det(A_2) = (2)(2) - (-1)(-1) = 3
    \end{equation}
    \item For $n=3$:
    \begin{equation}
    \begin{aligned}
        A_3 &= \begin{bmatrix} 2 & -1 & 0 \\ -1 & 2 & -1 \\ 0 & -1 & 2 \end{bmatrix} \\
        D_3 &= 2 \cdot \det\begin{pmatrix} 2 & -1 \\ -1 & 2 \end{pmatrix} - (-1) \cdot \det\begin{pmatrix} -1 & -1 \\ 0 & 2 \end{pmatrix} \\
        &= 2 D_2 + ((-1)(2) - (-1)(0)) = 2(3) - 2 = 4
    \end{aligned}
    \end{equation}
\end{itemize}
The sequence of determinants $D_1=2, D_2=3, D_3=4$ suggests the pattern $D_n = n+1$.

\textbf{Recurrence Relation}

We use cofactor expansion along the first row of $A_n$ to derive a general recurrence relation for $D_n = \det(A_n)$.
\begin{equation}
\begin{aligned}
D_n &= 2 \cdot \det
    \begin{pmatrix}
    2 & -1 & & \\
    -1 & 2 & \ddots \\
    & \ddots & \ddots & -1 \\
    & & -1 & 2
    \end{pmatrix}_{(n-1)\times(n-1)}
 - (-1) \cdot \det
    \begin{pmatrix}
    -1 & -1 & 0 & \dots \\
    0 & 2 & -1 & \\
    0 & -1 & 2 & \ddots \\
    \vdots & & \ddots & \ddots
    \end{pmatrix}_{(n-1)\times(n-1)}
\end{aligned}
\end{equation}
The first sub-determinant is simply $D_{n-1}$. For the second sub-determinant, we perform a cofactor expansion along its first column, which yields $-1 \cdot D_{n-2}$.
\begin{equation}
\begin{aligned}
D_n &= 2 D_{n-1} - (-1)(-1 \cdot D_{n-2}) \\
D_n &= 2 D_{n-1} - D_{n-2}
\end{aligned}
\end{equation}
This recurrence is valid for $n \ge 3$. We check if our hypothesized formula $D_n=n+1$ satisfies this recurrence.
\begin{equation}
2 D_{n-1} - D_{n-2} = 2((n-1)+1) - ((n-2)+1) = 2n - (n-1) = n+1 = D_n
\end{equation}
The formula holds for the base cases and satisfies the recurrence, so it is correct by induction.

\textbf{Final Result}

The given matrix $A$ has size $n = N-1$. Therefore, its determinant is:
\begin{equation}
\det(A) = D_{N-1} = (N-1) + 1 = N
\end{equation}

\section{Calculation of the Inverse of the Matrix}

Let $B = A^{-1}$. The $j$-th column of $B$, denoted by the vector $\bm{b}_j$, is the solution to the system $A \bm{b}_j = \bm{e}_j$, where $\bm{e}_j$ is the $j$-th standard basis vector. Writing this out for the $i$-th component (where $b_{ij}$ is the $(i,j)$-th element of $B$) gives the system of equations:
\begin{equation}
-b_{i-1, j} + 2b_{i, j} - b_{i+1, j} = \delta_{ij} \quad \text{for } i,j = 1, \dots, N-1
\end{equation}
with boundary conditions $b_{0,j} = 0$ and $b_{N,j} = 0$.

\textbf{Homogeneous Solution}

For $i \neq j$, the equation is homogeneous:
\begin{equation}
-b_{i-1, j} + 2b_{i, j} - b_{i+1, j} = 0
\end{equation}
This is a linear recurrence relation with characteristic equation $r^2 - 2r + 1 = 0$, or $(r-1)^2=0$. This has a repeated root $r=1$, so the general solution is linear in $i$:
\begin{equation}
b_{i,j} = C_1 + C_2 i
\end{equation}
We apply this solution to two regions.

\textbf{Region 1: $1 \le i \le j$}. The boundary condition $b_{0,j}=0$ implies $C_1+C_2(0)=0$, so $C_1=0$. The solution has the form:
\begin{equation}
b_{i,j} = C \cdot i
\end{equation}
\textbf{Region 2: $j \le i \le N-1$}. The boundary condition $b_{N,j}=0$ implies $D_1+D_2(N)=0$, so $D_1 = -D_2 N$. The solution is $b_{i,j} = -D_2 N + D_2 i = D_2(i-N)$. Letting $D = -D_2$, the solution has the form:
\begin{equation}
b_{i,j} = D \cdot (N-i)
\end{equation}

\textbf{Stitching the Solutions}

The full solution is given by:
\begin{equation}
b_{i,j} =
\begin{cases}
C \cdot i & \text{if } i \le j \\
D \cdot (N-i) & \text{if } i \ge j
\end{cases}
\end{equation}
The constants $C$ and $D$ are found by satisfying two conditions at $i=j$.
\begin{enumerate}
    \item Continuity at $i=j$: The two forms must be equal.
    \begin{equation}
    C \cdot j = D \cdot (N-j) \implies D = C \frac{j}{N-j}
    \end{equation}
    \item The inhomogeneous equation at $i=j$:
    \begin{equation}
    -b_{j-1, j} + 2b_{j, j} - b_{j+1, j} = 1
    \end{equation}
\end{enumerate}
Substituting the piecewise solutions into the inhomogeneous equation:
\begin{equation}
\begin{aligned}
-C(j-1) + 2(Cj) - D(N-(j+1)) &= 1 \\
C \left[ -(j-1) + 2j - \frac{j}{N-j}(N-j-1) \right] &= 1 \\
C \left[ j+1 - \frac{jN-j^2-j}{N-j} \right] &= 1 \\
C \left[ \frac{(j+1)(N-j) - (jN-j^2-j)}{N-j} \right] &= 1 \\
C \left[ \frac{jN-j^2+N-j - jN+j^2+j}{N-j} \right] &= 1 \\
C \left[ \frac{N}{N-j} \right] &= 1
\end{aligned}
\end{equation}
This gives the constants:
\begin{equation}
C = \frac{N-j}{N} \quad \text{and} \quad D = \left(\frac{N-j}{N}\right) \frac{j}{N-j} = \frac{j}{N}
\end{equation}

\textbf{Final Result}

The element $(i,j)$ of the inverse matrix $A^{-1}$ is:
\begin{equation}
(A^{-1})_{ij} =
\begin{cases}
\frac{N-j}{N} \cdot i & \text{if } i \le j \\
\frac{j}{N} \cdot (N-i) & \text{if } i \ge j
\end{cases}
\end{equation}
This can be written more compactly using $\min$ and $\max$ functions:
\begin{equation}
(A^{-1})_{ij} = \frac{\min(i,j) \cdot (N - \max(i,j))}{N}
\end{equation}