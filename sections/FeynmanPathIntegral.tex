\part{Feynman Path Integral}

\chapter{Feynman Path Integral}

The version of quantum mechanics:
\begin{enumerate}
    \item Schrödinger's wavefuction (operator form):
    \begin{equation}
        \hat{H}|\psi\rangle = \mathrm{i}\hbar\frac{\partial}{\partial t}|\psi\rangle
    \end{equation}
    \item Feynman's Path Integral (Common number form):
    \begin{equation}
        \mathrm{i}G(\text{Green's function}) \propto \int \mathcal{D}(x, t) \mathrm{e}^{\mathrm{i}\int \mathcal{L} \mathrm{d}t}
    \end{equation}
\end{enumerate}

There are many advantages of Feynman Path Integral:
\begin{enumerate}
    \item Make the double-slit experiment more understandable.
    \item The classical limit \enquote{$\hbar \to 0$} is \enquote{tractable}:
    quantum $\xrightarrow{\hbar \to 0}$ classical.
    \item Provide a semi-classical picture. for. quantum mechanics.
    \item \enquote{Quantum fluctuations} are more \enquote{understandable}.
    \item A natural route. to low energy effective theory of quantum many-body systems.
    \item A natural language for describing topological properties of quantum many-body systems.
\end{enumerate}

But the practical calculation in the path-integral representation
of simple quantum mechanical problem many be notoriously
difficult and lengthy.

\section{Propagators}
Consider a quantum partical confined in a one-dimensional space:
\begin{equation}
\hat{H} = \frac{\hat{p}^2}{2m} + V(\hat{x})
\end{equation}
and the canonical pair: $[\hat{x}, \hat{p}] = \hat{x}\hat{p} - \hat{p}\hat{x} = \mathrm{i}\hbar$

The Schrödinger's equation is:
\begin{equation}
\mathrm{i}\hbar \frac{\partial |\psi(t)\rangle}{\partial t} = \hat{H}|\psi(t)\rangle
\end{equation}

This first-order nature allows us to define a time evolution
operator $\hat{U}(t, t_0)$ which propagates the state vector from
an initial time $t_0$ to a final time $t$:
\begin{equation}
|\psi(t)\rangle = \hat{U}(t, t_0)|\psi(t_0)\rangle
\end{equation}

Assuming the Hamiltonian $H$ is not explicitly depend on
time, the formal solution of $\hat{U}$ is:
\begin{equation}
\hat{U} = \mathrm{e}^{-\frac{\mathrm{i}}{\hbar}\hat{H}(t-t_0)}
\end{equation}

A crucial property of $\hat{U}$ is the \enquote{chain-like} rule, or composition property. For any intermediate time $t'$ such that $t>t'>t_0$:
\begin{equation}
\hat{U}(t, t_0) = \hat{U}(t, t') \hat{U}(t', t_0)
\end{equation}

This property is the key to the entire path integral derivation.
And $\hat{U}(t, t_0)$ is unitary:
\begin{equation}
\hat{U}^{\dagger}\hat{U} = \hat{U}\hat{U}^{\dagger} = \hat{\mathbb{I}}
\end{equation}
where $\hat{\mathbb{I}}$ is the identity operator.

In the position representation, we can obtain matrix elements:
\begin{equation}
\begin{aligned}
U(x, t; x_0, t_0) &= \langle x | \hat{U}(t, t_0) | x_0 \rangle \\
&= \langle x | \mathrm{e}^{-\frac{\mathrm{i}}{\hbar}\hat{H}(t-t_0)} | x_0 \rangle
\end{aligned}
\end{equation}

We can define a propagators (Green's function) of the quantum system by using the matrix elements:
\begin{equation}
\mathrm{i}G(x, t; x_0, t_0) = U(x, t; x_0, t_0)
\end{equation}

Using the matrix elements, $\psi(x, t)$ can be reformulated as:
\begin{equation}
\begin{aligned}
\psi(x, t) &= \langle x | \hat{U}(t, t_0) | \psi(t_0) \rangle \\
&= \int \mathrm{d}x_0 \langle x | \hat{U}(t, t_0) | x_0 \rangle \langle x_0 | \psi(t_0) \rangle \\
&= \int \mathrm{d}x_0 U(x, t; x_0, t_0) \psi(x_0, t_0)
\end{aligned}
\end{equation}

Also, the propagator also satisfies the Schrödinger's equation:
\begin{equation}
    \colorboxed{\mathrm{i}\hbar \frac{\partial}{\partial t} G(x, t; x_0, t_0) = \hat{H} G(x, t; x_0, t_0)}
\end{equation}

And the initial condition is:
\begin{equation}
G(x, t_0; x_0, t_0) = -\mathrm{i} \langle x | \hat{U}(t_0, t_0) | x_0 \rangle = -\mathrm{i} \delta(x-x_0)
\end{equation}
$\delta(x-x_0)$ is Dirac function.

\begin{example}

    For free particle: $\hat{H} = \frac{1}{2m}\hat{p}^2$, in position representation:
    \begin{equation}
    \hat{p} \to -\mathrm{i}\hbar \frac{\partial}{\partial x}
    \end{equation}

    So the PDE is:
    \begin{equation}
    \mathrm{i}\hbar \frac{\partial}{\partial t} G(x, t; x_0, t_0) = -\frac{\hbar^2}{2m} \frac{\partial^2}{\partial x^2} G(x, t; x_0, t_0)
    \end{equation}

    Solve the PDE:

    Use Fourier Transform: (we use $G(x,t)$ instead of $G(x,t;x_0,t_0)$). We solve the free-particle Green's function by transforming to momentum space:
    \begin{equation}
        \left\{
        \begin{aligned}
        G(x, t) &= \frac{1}{\sqrt{2\pi}} \int \mathrm{d}k \cdot \tilde{G}(k,t) \mathrm{e}^{\mathrm{i}kx} \\
        \tilde{G}(k, t) &= \frac{1}{\sqrt{2\pi}} \int \mathrm{d}x \ G(x,t) \mathrm{e}^{-\mathrm{i}kx}
        \end{aligned}
        \right.
    \end{equation}

    With these conventions, spatial derivatives become algebraic in $k$-space while the time derivative remains unchanged:
    \begin{equation}
        \left\{
        \begin{aligned}
        \mathcal{F}\left\{ \mathrm{i}\hbar \frac{\partial G(x,t)}{\partial t} \right\} &= \mathrm{i}\hbar \frac{\partial \tilde{G}(k,t)}{\partial t} \\
        \mathcal{F}\left\{ -\frac{\hbar^2}{2m} \frac{\partial^2 G}{\partial x^2} \right\} &= -\frac{\hbar^2}{2m} [ -k^2 \tilde{G}(k,t) ] = \frac{\hbar^2 k^2}{2m} \tilde{G}(k,t)
        \end{aligned}
        \right.
    \end{equation}

    Applying the transform to the PDE yields an ordinary differential equation in time for each $k$:
    \begin{equation}
        \left\{
        \begin{aligned}
        &\mathrm{i}\hbar \frac{\partial}{\partial t} \tilde{G}(k,t) = \frac{\hbar^2 k^2}{2m} \tilde{G}(k,t)\\
        &\frac{\mathrm{d}\tilde{G}}{\tilde{G}} = -\mathrm{i} \frac{\hbar k^2}{2m} \mathrm{d}t
        \end{aligned}
        \right.
    \end{equation}

    Integrating in time gives the logarithm of the solution up to a $k$-dependent constant:
    \begin{equation}
    \ln \tilde{G} = -\mathrm{i} \frac{\hbar k^2}{2m} t + C(k)
    \end{equation}

    So we can get the solution:
    \begin{equation}
    \tilde{G}(k,t) = A(k) \mathrm{e}^{-\mathrm{i}\frac{\hbar k^2}{2m}t}, \quad A(k) = \mathrm{e}^{C(k)}
    \end{equation}

    To determine $A(k)$, impose the initial condition at time $t_0$ in position space:
    \begin{equation}
    \begin{aligned}
    \tilde{G}(k, t_0) &= \frac{1}{\sqrt{2\pi}} \int_{-\infty}^{\infty} \mathrm{d}x G(x, t_0) \mathrm{e}^{-\mathrm{i}kx} \\
    &= \frac{1}{\sqrt{2\pi}} \int_{-\infty}^{\infty} \mathrm{d}x \delta(x-x_0) \mathrm{e}^{-\mathrm{i}kx}
    \end{aligned}
    \end{equation}

    Using the Fourier transform of the Dirac delta, we find:
    \begin{equation}
    \tilde{G}(k, t_0) = \frac{1}{\sqrt{2\pi}} \mathrm{e}^{-\mathrm{i}kx_0}
    \end{equation}

    Matching at $t_0$ fixes the $k$-space amplitude:
    \begin{equation}
    A(k) = \frac{1}{\sqrt{2\pi}} \mathrm{e}^{-\mathrm{i}kx_0} \cdot \mathrm{e}^{\mathrm{i}\frac{\hbar k^2}{2m}t_0}.
    \end{equation}

    Therefore, for general time $t$ we have:
    \begin{equation}
    \tilde{G}(k,t) = \frac{1}{\sqrt{2\pi}} \mathrm{e}^{-\mathrm{i}kx_0} \mathrm{e}^{-\mathrm{i}\frac{\hbar k^2}{2m}(t-t_0)}
    \end{equation}

    Finally, inverse-transform back to position space to obtain the integral representation of the propagator:
    \begin{equation}
    \begin{aligned}
    G(x,t) &= \frac{1}{\sqrt{2\pi}} \int_{-\infty}^{\infty} \mathrm{d}k \cdot \tilde{G}(k,t) \mathrm{e}^{\mathrm{i}kx} \\
    &= \frac{1}{2\pi} \int_{-\infty}^{\infty} \mathrm{d}k \, \mathrm{e}^{\mathrm{i}k(x-x_0)} \cdot \mathrm{e}^{-\mathrm{i}\frac{\hbar(t-t_0)}{2m}k^2}
    \end{aligned}
    \end{equation}

    This is a standard Gaussian integral of the form:
    \begin{equation}
    \int_{-\infty}^{\infty} \mathrm{d}k \, \mathrm{e}^{-ak^2+bk} = \sqrt{\frac{\pi}{a}} \mathrm{e}^{\frac{b^2}{4a}}.
    \end{equation}

    Let's identify the coefficients:
    \begin{equation}
    \left\{
    \begin{aligned}
    &a = \mathrm{i} \frac{\hbar(t-t_0)}{2m} \\
    &b = \mathrm{i}(x-x_0)
    \end{aligned}
    \right.
    \end{equation}

    So we can get the solution:
    \begin{equation}
        \mathrm{i}G(x,t) 
        = \left[ \frac{m}{2\pi\hbar\mathrm{i}(t-t_0)} \right]^{\frac{1}{2}} \cdot \mathrm{e}^{\frac{\mathrm{i}}{\hbar} 
        \cdot \frac{m(x-x_0)^2}{2(t-t_0)}}
    \end{equation}

    Also, we can solve this PDE via definition:
    \begin{equation}
    \begin{aligned}
    \mathrm{i}G &= \langle x | \mathrm{e}^{-\frac{\mathrm{i}}{\hbar}(t-t_0)\hat{H}} | x_0 \rangle \\
    &= \int \langle x | \mathrm{e}^{-\frac{\mathrm{i}}{\hbar}(t-t_0)\hat{H}} | p \rangle \langle p | x_0 \rangle \, \mathrm{d}p \\
    &= \int \mathrm{d}p \, \mathrm{e}^{-\frac{\mathrm{i}(t-t_0)p^2}{2m\hbar}} \langle x | p \rangle \langle p | x_0 \rangle \\
    &= \frac{1}{2\pi\hbar} \int \mathrm{d}p \, \mathrm{e}^{-\frac{\mathrm{i}(t-t_0)p^2}{2m\hbar} + \mathrm{i}\frac{(x-x_0)}{\hbar}p}.
    \end{aligned}
    \end{equation}
    we use $P = \hbar k$ and can get the same equation as
    the Fourier Transform Method.

\end{example}


\section{Path-Integral}

When $t > t_1 > t_0$, and $t_1$ is an arbitrarily selected intermediate time,
we can write:
\begin{equation}
    \begin{aligned}
    \mathrm{i}G(x, t; x_0, t_0) &= \langle x | \hat{U}(t, t_0) | x_0 \rangle \\
    &= \langle x | \hat{U}(t, t_1) \hat{U}(t_1, t_0) | x_0 \rangle \\
    &= \int \mathrm{d}x_1 \langle x | \hat{U}(t, t_1) | x_1 \rangle \langle x_1 | \hat{U}(t_1, t_0) | x_0 \rangle \\
    &= \int \mathrm{d}x_1 \, \mathrm{i}G(x, t; x_1, t_1) \cdot \mathrm{i}G(x_1, t_1; x_0, t_0)
    \end{aligned}
\end{equation}

This intergral over $x_1$ means \enquote{superposition} of all possible \enquote{path} that connect $x$ and $x_0$.
Next, we try to \enquote{smooth} the path along time directly.
We can insert more time slices between $x$ and $x_0$.
If we insert infinite time slices, the path become smooth.

\begin{center}
\begin{figure}[htbp]
    \centering
    \begin{tikzpicture}[
        >=Stealth,
        % Define global styles
        axis/.style={->, thick},
        axis label/.style={font=\itshape},
        point label/.style={font=\scriptsize},
        slice line/.style={dashed, green!60!black, thin},
        classical/.style={red, thick},
        quantum/.style={black!70, thin},
        % Style for arrows in the middle of paths
        mid arrow/.style={postaction={decorate}, decoration={
            markings,
            mark=at position 0.55 with {\arrow{>}}
        }},
        % Style for the big block arrows connecting diagrams
        block arrow/.style={
            single arrow, draw=black, fill=white, 
            minimum height=1.5cm, minimum width=1.5cm,
            single arrow head extend=0.3cm,
            drop shadow
        }
    ]
        % -----------------------------------------------------
        % STAGE 1: ONE TIME SLICE (Top Left)
        % -----------------------------------------------------
        \begin{scope}[local bounding box=stage1]
            \coordinate (O1) at (0,0);
            \coordinate (Start1) at (0.5, 0.5);
            \coordinate (End1) at (3.5, 3.5);
            
            % Axes
            \draw[axis] (0,0) -- (4.5,0) node[right, axis label] {\small Space};
            \draw[axis] (0,0) -- (0,4) node[above, axis label] {\small Time};
            
            % Time Slices (N=1)
            \draw[slice line] (0, 2) -- (4, 2);
            \node[green!60!black, font=\scriptsize, fill=black!5, rounded corners, inner sep=1pt] at (2.5, 2) {time slice};
            
            % Projections
            \draw[dashed, gray] (Start1) -- (0, 0.5) node[left, point label] {$t_c$};
            \draw[dashed, gray] (End1) -- (0, 3.5) node[left, point label] {$t_a$};

            % Paths (Piecewise linear with 1 kink)
            % Classical (Straight)
            \draw[classical, mid arrow] (Start1) -- (End1);
            
            % Quantum (Kinked at slice)
            % Path A (Left side)
            \draw[quantum, mid arrow] (Start1) -- (0.2, 2) -- (End1);
            % Path B (Right side)
            \draw[quantum, mid arrow] (Start1) -- (3.8, 2) -- (End1);
            % Path C (Small kink)
            \draw[quantum, mid arrow] (Start1) -- (2.5, 2) -- (End1);

            % Points
            \filldraw[black] (Start1) circle (1.5pt) node[below, point label] {$x_c$};
            \filldraw[white, draw=black, thick] (End1) circle (1.5pt);
        \end{scope}

        % -----------------------------------------------------
        % STAGE 2: THREE TIME SLICES (Top Right)
        % -----------------------------------------------------
        \begin{scope}[xshift=7cm, local bounding box=stage2]
            \coordinate (Start2) at (0.5, 0.5);
            \coordinate (End2) at (3.5, 3.5);
            
            % Axes
            \draw[axis] (0,0) -- (4.5,0) node[right, axis label] {\small Space};
            \draw[axis] (0,0) -- (0,4) node[above, axis label] {\small Time};
            
            % Time Slices (N=3)
            \foreach \y in {1.25, 2.0, 2.75} {
                \draw[slice line] (0, \y) -- (4, \y);
            }
            \node[green!60!black, font=\scriptsize, draw, rounded corners] at (3.5, 2.0) {3 slices};
            
            % Paths (Piecewise linear with 3 kinks)
            % Classical
            \draw[classical, mid arrow] (Start2) -- (End2);
            
            % Quantum Path A (Wide zig-zag)
            \draw[quantum, mid arrow] (Start2) -- (0.2, 1.25) -- (0.5, 2.0) -- (0.8, 2.75) -- (End2);
            % Quantum Path B (Other side)
            \draw[quantum, mid arrow] (Start2) -- (3.5, 1.25) -- (3.0, 2.0) -- (3.8, 2.75) -- (End2);
            % Quantum Path C (Near classical)
            \draw[quantum, mid arrow] (Start2) -- (1.5, 1.25) -- (1.8, 2.0) -- (2.5, 2.75) -- (End2);

            % Points
            \filldraw[black] (Start2) circle (1.5pt) node[below, point label] {$x_c$};
            \filldraw[white, draw=black, thick] (End2) circle (1.5pt);
        \end{scope}

        % -----------------------------------------------------
        % STAGE 3: INFINITE SLICES / SMOOTH (Bottom Center)
        % -----------------------------------------------------
        \begin{scope}[xshift=3.5cm, yshift=-6.5cm, local bounding box=stage3]
            \coordinate (Start3) at (0.5, 0.5);
            \coordinate (End3) at (3.5, 3.5);
            
            % Axes
            \draw[axis] (0,0) -- (5.5,0) node[right, axis label] {Space ($x$)};
            \draw[axis] (0,0) -- (0,4.5) node[above, axis label] {Time ($t$)};
            
            % Slices (Implied/Many) - represented by text
            \node[align=left, text width=3cm, draw=black, rounded corners, inner sep=4pt] 
                at (6.5, 2.5) 
                {\textbf{``Infinite'' slices} \\ \scriptsize (Superposition)};
            
            % Paths (Smooth Curves)
            % Classical
            \draw[classical, mid arrow] (Start3) -- (End3) node[midway, above left, red, font=\tiny] {Classical};
            
            % Quantum (Smooth Bezier)
            \draw[quantum, mid arrow] (Start3) .. controls (4.5, 1.5) and (4.5, 2.5) .. (End3);
            \draw[quantum, mid arrow] (Start3) .. controls (3.0, 1.5) and (3.5, 3.0) .. (End3);
            \draw[quantum, mid arrow] (Start3) .. controls (-1.5, 1.5) and (-1.0, 3.0) .. (End3);
            \draw[quantum, mid arrow] (Start3) .. controls (0.0, 2.0) and (1.5, 3.0) .. (End3);
            \draw[quantum, mid arrow] (Start3) .. controls (1.5, 2.0) and (2.0, 1.5) .. (End3);

            % Points
            \filldraw[black] (Start3) circle (1.5pt) node[below, point label] {$x_c$};
            \filldraw[white, draw=black, thick] (End3) circle (1.5pt) node[left=2pt, point label] {$t_a$} node[below=2pt, point label] {$x_a$};
            
            % Projections
            \draw[dashed, gray] (Start3) -- (0, 0.5) node[left, point label] {$t_c$};
            \draw[dashed, gray] (End3) -- (0, 3.5);

        \end{scope}

        % -----------------------------------------------------
        % TRANSITION ARROWS
        % -----------------------------------------------------
        % Arrow 1: Stage 1 -> Stage 2
        \node[block arrow, rotate=0, fill=black!10] at (5.75, 2) {\scriptsize Refine};
        
        % Arrow 2: Stage 1/2 -> Stage 3 (Pointing down/center)
        \node[block arrow, rotate=-90, fill=black!10, minimum height=1.2cm] at (5.75, -1.2) {\scriptsize $N \to \infty$};

    \end{tikzpicture}
    \caption{Evolution of the path integral formulation: (Top Left) A single time slice yields coarse, kinked paths. (Top Right) Adding more slices ($N=3$) refines the grid. (Bottom) Taking the limit of infinite time slices ($N \to \infty$) recovers the smooth paths of standard quantum mechanics.}
    \label{fig:smoothing_path_full}
\end{figure}
\end{center}

Firstly, let's discretize time. domain $[t_0, t]$ into $N$ pieces of equal length $\Delta t = \frac{t-t_0}{N}$:
\begin{equation}
\begin{aligned}
\mathrm{i}G(x, t; x_0, t_0) &= \langle x | \hat{U}(t, t_{N-1}) \hat{U}(t_{N-1}, t_{N-2}) \cdots \hat{U}(t_1, t_0) | x_0 \rangle \\
&= \int \mathrm{d}x_{N-1} \cdots \mathrm{d}x_1 \prod_{l=1}^{N} \mathrm{i}G(x_l, t_l; x_{l-1}, t_{l-1})
\end{aligned}
\end{equation}

let $\mathcal{D}_x = \prod_{l=1}^{N-1} \mathrm{d}x_l$.
Consider $N \to \infty$, so $\Delta t = \frac{t-t_0}{N} \to 0$, which means $t_l - t_{l-1} = \Delta t$.
\begin{equation}
    \label{eq:path_integral}
    \mathrm{i}G(x_l, t_l; x_{l-1}, t_{l-1}) = \langle x_l | \mathrm{e}^{-\frac{\mathrm{i}}{\hbar}\hat{H}\Delta t} | x_{l-1} \rangle
\end{equation}

Because $\Delta t$ is small, we can approximate the exponential function by its Taylor series:
\begin{equation}
    \label{eq:exp_approx}
    \mathrm{e}^{-\frac{\mathrm{i}}{\hbar}\hat{H}\Delta t} \approx \hat{\mathbb{I}} - \frac{\mathrm{i}}{\hbar}\hat{H}\Delta t 
    = \hat{\mathbb{I}} - \frac{\mathrm{i}}{\hbar}\Delta t \left[ \frac{\hat{p}^2}{2m} + V(\hat{x}) \right]
\end{equation}

Substitute \eqref{eq:exp_approx} into \eqref{eq:path_integral}:
\begin{equation}
\begin{aligned}
\mathrm{i}G(x_l, t_l; x_{l-1}, t_{l-1}) &= \int \mathrm{d}p_l \langle x_l | p_l \rangle \langle p_l | \hat{\mathbb{I}} - \frac{\mathrm{i}}{\hbar}\Delta t \left[ \frac{\hat{p}^2}{2m} + V(\hat{x}) \right] | x_{l-1} \rangle \\
&= \int \mathrm{d}p_l \langle x_l | p_l \rangle \langle p_l | x_{l-1} \rangle \left[ 1 - \frac{\mathrm{i}}{\hbar} \left( \frac{p_l^2}{2m} + V(x_{l-1}) \right) \Delta t \right] \\
\end{aligned}
\end{equation}

With the approximations $V(x_l) \approx V(x_{l-1})$:
\begin{equation}
\left[ 1 - \frac{\mathrm{i}}{\hbar} \left( \frac{p_l^2}{2m} + V(x_{l-1}) \right) \Delta t \right] 
\approx \left( 1 - \frac{\mathrm{i}}{\hbar} H_{l} \Delta t \right) \approx \mathrm{e}^{-\frac{\mathrm{i}}{\hbar}H_{l}\Delta t}
\end{equation}

So $\mathrm{i}G(x_l, t_l; x_{l-1}, t_{l-1})$ can be written as:
\begin{equation}
    \begin{aligned}
    \mathrm{i}G(x_l, t_l; x_{l-1}, t_{l-1}) &= \int \mathrm{d}p_l \frac{1}{2\pi\hbar} \mathrm{e}^{\frac{\mathrm{i}}{\hbar}p_l(x_l-x_{l-1})} \mathrm{e}^{-\frac{\mathrm{i}}{\hbar}H_{cl}\Delta t} \\
    &= \frac{1}{2\pi\hbar} \int \mathrm{d}p_l e^{\frac{\mathrm{i}}{\hbar}[p_l(x_l-x_{l-1})-H_{l}\Delta t]} \\
    &= \frac{1}{2\pi\hbar} \int \mathrm{d}p_l \mathrm{e}^{\frac{\mathrm{i}}{\hbar}[p_l(\frac{x_l-x_{l-1}}{\Delta t})-H_{l}]\Delta t}
    \end{aligned}
\end{equation}
where, $H_l$ is the classical Hamiltonion as a function of $p_l$ and $x_l$.

When $\Delta t \to 0$:
\begin{equation}
\frac{x_l - x_{l-1}}{\Delta t} = \dot{x}_l
\end{equation}

So we can get:
\begin{equation}
p_l\dot{x}_l - H_l = \mathcal{L}_l.
\end{equation}
where, $\mathcal{L}_l$ is the classical Lagrangian.

So $\mathrm{i}G(x_l, t_l; x_{l-1}, t_{l-1})$ can be written as the form with Lagrangian:
\begin{equation}
\mathrm{i}G(x_l, t_l; x_{l-1}, t_{l-1}) = \int \mathrm{d}p_l \cdot \frac{1}{2\pi\hbar} \mathrm{e}^{\frac{\mathrm{i}}{\hbar}\mathcal{L}_l\Delta t}.
\end{equation}

Substitute $\mathrm{i}G(x_l, t_l; x_{l-1}, t_{l-1})$ into the path integral:
\begin{equation}
\prod_{l=1}^{N} \mathrm{i}G(x_l, t_l; x_{l-1}, t_{l-1}) = \int \frac{\mathrm{d}p_N}{2\pi\hbar} \cdots \frac{\mathrm{d}p_1}{2\pi\hbar} \cdot \mathrm{e}^{\frac{\mathrm{i}}{\hbar}\sum_{l=1}^{N}\mathcal{L}_l \cdot \Delta t}
\end{equation}

let $\mathcal{D}_p = \prod_{l=1}^{N} \frac{\mathrm{d}p_l}{2\pi\hbar}$, when $\Delta t \to 0$, which means:
\begin{equation}
\sum_{l=1}^{N} \mathcal{L}_l \Delta t = \int_{t_0}^{t} \mathrm{d}\tau \cdot \mathcal{L}[p(\tau), x(\tau)]
\end{equation}

Finally, we can get the propagators by the path integral:
\begin{theorem}
    The propagators path integral:
    \begin{equation}
        \mathrm{i}G(x, t; x_0, t_0) 
        = \int \mathcal{D}_x \mathcal{D}_p \cdot \mathrm{e}^{\frac{\mathrm{i}}{\hbar}\int_{t_0}^{t} \mathrm{d}\tau \cdot \mathcal{L}[p(\tau), x(\tau)]}
    \end{equation}
    where, the pair of $p(t)$ and $\dot{x}(t)$ characterizes a path in the px phase space.
\end{theorem}

\begin{center}
\begin{figure}[htbp]
    \centering
    \begin{tikzpicture}[
        scale=1.5,
        >=Stealth,
        % Style for placing an arrow in the middle of a path
        midarrow/.style={
            decoration={
                markings,
                mark=at position 0.6 with {\arrow[scale=1.2]{>}}
            },
            postaction={decorate}
        },
        axis/.style={->, thick, draw=black!80},
        label node/.style={font=\small\itshape}
    ]

        % --- COORDINATES ---
        \coordinate (Origin) at (0,0); % (tc, xc)
        \coordinate (Target) at (3,4); % Destination point

        % --- AXES ---
        % Vertical Axis (Time)
        \draw[axis] (0, -0.5) -- (0, 5.5) node[left] {\Large Time};
        % Horizontal Axis (Space)
        \draw[axis] (-0.5, 0) -- (5, 0) node[below] {\Large Space};

        % --- TICKS AND LABELS ---
        % Origin labels (tc, xc)
        \node[below left] at (Origin) {\large $t_c, x_c$};
        \filldraw[black] (Origin) circle (1.5pt);

        % Target labels (ta, xa) projections
        \draw[dashed, gray] (Target) -- (0, 4) node[left, black] {\large $t_a$};
        \draw[dashed, gray] (Target) -- (3, 0) node[below, black] {\large $x_a$};
        \draw[dashed, gray] (0,0) -- (0,0); % Dummy for anchor

        % --- PATHS ---

        % 1. Quantum Paths (The "Sum over histories")
        % Using Bezier curves to simulate random paths
        
        % Path Left 1
        \draw[thick, midarrow, gray!80] (Origin) .. controls (-1, 1) and (-0.5, 3) .. (Target);
        
        % Path Left 2 (Wider)
        \draw[thick, midarrow, gray!80] (Origin) .. controls (-2, 1.5) and (-1, 3.5) .. (Target);
        
        % Path Right 1
        \draw[thick, midarrow, gray!80] (Origin) .. controls (1.5, 0.5) and (2.5, 2) .. (Target);
        
        % Path Right 2 (Wider loop)
        \draw[thick, midarrow, gray!80] (Origin) .. controls (4, 1) and (4.5, 3) .. (Target);
        
        % Path Middle Wiggle
        \draw[thick, midarrow, gray!80] (Origin) .. controls (0.5, 2) and (2.5, 2) .. (Target);

        % Path Cross-over
        \draw[thick, midarrow, gray!80] (Origin) .. controls (1, 3) and (-1, 2) .. (Target);

        % 2. Classical Path (The Red Line - Newton's Law)
        % Drawn last to be on top
        \draw[red, very thick, midarrow] (Origin) -- (Target);
        
        % Target Point Dot
        \filldraw[white, draw=black, thick] (Target) circle (2pt);

        % --- ANNOTATION TEXT ---
        % "This red path is 'classical path'..."
        \node[align=left, text=red!80!black, font=\small] (annotation) at (5.5, 4.5) {
            This red path is ``classical path'' \\
            for free particle \\
            (Newton's law)
        };

        % Arrow from annotation to the red path
        \draw[->, red!80!black, dashed, thick] (annotation.west) to[bend right=20] (1.6, 2.2);

    \end{tikzpicture}
    \caption{Schematic of Feynman Path Integral. The red path is the classical path (Newton's law) for a free particle, while the gray paths represent the quantum sum over histories.}
    \label{fig:path_integral}
\end{figure}
\end{center}

\section{Gaussian Integration}

If the functional integration over p is Gaussian, we can exactly integrate out p. For example, $H = \frac{p^2}{2m} + V$,
so $\mathcal{L} = p\dot{x} - H = p\dot{x} - \frac{p^2}{2m} - V(x)$,we can get:

\begin{equation}
\mathrm{i}G = \int \mathcal{D}p \mathcal{D}x \exp \left[ \frac{\mathrm{i}}{\hbar} \sum_t \left( p_t \dot{x}_t - \frac{p_t^2}{2m} - V(x_t) \right) \Delta t \right]
\end{equation}
Let:
\begin{equation}
\bm{p} = \begin{pmatrix} p_l \\ \vdots \\ p_1 \end{pmatrix}, \quad \dot{\bm{x}} = \begin{pmatrix} \dot{x}_l \\ \vdots \\ \dot{x}_1 \end{pmatrix}
\end{equation}
So we can rewrite the integral as:
\begin{equation}
\mathrm{i}G = \int \mathcal{D}x \cdot \exp \left[ \frac{\mathrm{i}}{\hbar} \sum_{l = 1}^{N} (-V(x_l)) \Delta t \right] \cdot 
\int \mathcal{D}p \cdot \exp \left[ \frac{\mathrm{i}}{2m\hbar} (-\bm{p}^T \bm{p} + 2m \bm{p}^T \dot{\bm{x}}) \Delta t \right]
\end{equation}

We have an useful formula for Gaussian integral (Proof in \ref{appendix:gaussian-integral}):
\begin{equation}
    \colorboxed{
    \int \prod_{n = 1}^{N} \mathrm{d}x_n \exp \left[ -\frac{1}{2}\bm{x}^T A \bm{x} - \bm{x}^T \bm{y} \right] 
    = (2\pi)^{\frac{N}{2}} (\det A)^{-\frac{1}{2}} \exp \left[ \frac{1}{2} \bm{y}^T A^{-1} \bm{y} \right]
    }
\end{equation}
where, $\bm{x}, \bm{y}$ are real vectors and $A$ is real symmetric matrix.

Let:
\begin{equation}
    \begin{aligned}
    I &= \int \mathcal{D}p \exp \left[ \frac{\mathrm{i}}{2m\hbar} (-\bm{p}^T \bm{p} + 2m \bm{p}^T \dot{\bm{x}})\Delta t \right] \\
    &= \left( \frac{1}{2\pi\hbar} \right)^N \int \prod_{n = 1}^{N} \mathrm{d}p_n \cdot \exp \left[ -\frac{1}{2} \bm{p}^T A \bm{p} 
    - \bm{p}^T \dot{\bm{x}}' \right]
    \end{aligned}
\end{equation}
where $A = \frac{\mathrm{i}\Delta t}{m\hbar} \mathbb{I}_{N \times N}$ and $ \dot{\bm{x}}' = -\frac{\mathrm{i}\Delta t}{\hbar} \dot{\bm{x}}$, 
$\mathbb{I}_{N \times N}$ is the $N \times N$ identity matrix.

So we can get:
\begin{equation}
\left\{
\begin{aligned}
&(\det A)^{-\frac{1}{2}} = \left( \frac{\mathrm{i}\Delta t}{m\hbar} \right)^{-\frac{N}{2}}\\
&A^{-1} = \frac{m\hbar}{\mathrm{i}\Delta t} \mathbb{I}_{N \times N}
\end{aligned}
\right.
\end{equation}

The exponent term is:
\begin{equation}
\begin{aligned}
\frac{1}{2} \dot{\bm{x}}'^T A^{-1} \dot{\bm{x}}' 
&= \frac{1}{2} \cdot \frac{m\hbar}{\mathrm{i}\Delta t} \cdot \left( -\frac{\mathrm{i} \Delta t}{\hbar} \right)^2 \sum_{l=1}^N \dot{x}_l^2 \\
&= \frac{\mathrm{i}}{\hbar} \sum_{l=1}^N \frac{m}{2} (\frac{x_l - x_{l-1}}{\Delta t})^2 \Delta t
\end{aligned}
\end{equation}

So:
\begin{equation}
\begin{aligned}
I &= \left( \frac{1}{2\pi\hbar} \right)^N \cdot (2\pi)^{\frac{N}{2}} \cdot \left( \frac{\mathrm{i}\Delta t}{m\hbar} \right)^{-\frac{N}{2}} 
\cdot \exp \left[ \frac{\mathrm{i}}{\hbar} \sum_{l=1}^N  \frac{m}{2} (\frac{x_l-x_{l-1}}{\Delta t})^2 \Delta t \right] \\
&= \left( \frac{m}{\mathrm{i} 2\pi\hbar\Delta t} \right)^{\frac{N}{2}} 
\exp \left[ \frac{\mathrm{i}}{\hbar} \sum_{l=1}^N \frac{m}{2} (\frac{x_l-x_{l-1}}{\Delta t})^2 \Delta t \right]
\end{aligned}
\end{equation}

So we can get the integration without $p$:
\begin{equation}
    \begin{aligned}
    \mathrm{i}G &= \left( \frac{m}{\mathrm{i} 2\pi\hbar\Delta t} \right)^{\frac{N}{2}} 
    \int \mathcal{D}x \exp \left[ \frac{\mathrm{i}}{\hbar} \sum_{l=1}^N  \frac{m}{2} (\frac{x_l-x_{l-1}}{\Delta t})^2 \Delta t - V(x_l) \Delta t \right] \\
    &= \left( \frac{m}{\mathrm{i} 2\pi\hbar\Delta t} \right)^{\frac{N}{2}} \int \mathcal{D}x 
    \exp \left[ \frac{\mathrm{i}}{\hbar} \int_{t_0}^{t} \mathrm{d} \tau \, \mathcal{L}(x, \dot{x}) \right]
    \end{aligned}
\end{equation}

So the path-integral is proportional to :
\begin{equation}
    \colorboxed{
    \mathrm{i}G \propto \int \mathcal{D}x \cdot e^{\frac{\mathrm{i}}{\hbar}S[x(t)]}
    }
\end{equation}
where, $S[x(t)] = \int_{t_0}^{t} \mathrm{d}t \, \mathcal{L}(x, \dot{x})$ is action
and $\mathcal{L}(x, \dot{x}) = \frac{m\dot{x}^2}{2} - V(x)$ is Lagrangean.

From the Path-integral in real space-time, we can get some information about Physics Picture:
\begin{enumerate}[(1)]
    \item Each path is weighted with a $U(1)$ phase factor $e^{\frac{\mathrm{i}}{\hbar}S}$.
    The Quantum interference effect between different paths.
    \item Since $\hbar \sim 10^{-34} \, \mathrm{J \cdot s}$, any "small change" in S (we change S to $S+\delta S$), 
    will drastically lead to quantum destructive interference. 
    So only the paths that satisfy $\delta S=0$ make dominant contributions to the path-integral.
    \item Remarkably, $\delta S=0$ is exactly Hamilton's Principle in classical mechanics. 
    So the classical paths ($\delta S=0$) dominate the path integral in the limit $\hbar \to 0$. In other words, 
    in classical mechanics, as $\hbar \to 0$, 
    it neglects the contribution of the integral over all other paths near the path with $\delta S=0$. 
    So we can get the conclusion:
    \begin{equation*}
    \text{Quantum system} \xrightarrow{\hbar \to 0} \text{classical system}
    \end{equation*}
\end{enumerate}

\begin{example}
    Free particles' Hamiltonian is $H = \frac{p^2}{2m}$.\

    Using this Hamiltonian, we can get the path-integral:
    \begin{equation}
    \mathrm{i}G = \left( \frac{m}{i 2\pi\hbar\Delta t} \right)^{\frac{N}{2}} 
    \int \mathcal{D}x \exp\left[\frac{\mathrm{i}}{\hbar} \sum_{l=1}^N \frac{m}{2} (\frac{x_l-x_{l-1}}{\Delta t})^2 \Delta t \right].
    \end{equation}

    we let:
    \begin{equation}
    I = \int \mathcal{D}x \exp\left[\frac{\mathrm{i}m}{2\hbar\Delta t} \sum_{l=1}^N (x_l^2 - 2x_l x_{l-1} + x_{l-1}^2)\right]
    \end{equation}

    In order to use Gaussian integral:
    \begin{equation}
        \colorboxed{
        \int \prod_{n=1}^N \mathrm{d}x_n \, e^{-\frac{1}{2}\bm{x}^T A \bm{x} - \bm{x}^T \bm{y}} 
        = (2\pi)^{\frac{N}{2}} (\det A)^{-\frac{1}{2}} e^{\frac{1}{2}\bm{y}^T A^{-1} \bm{y}}
        }
    \end{equation}
    we should rewrite the form of $\exp[\frac{\mathrm{i}m}{2\hbar\Delta t} \sum (x_l^2-2x_l x_{l-1} + x_{l-1}^2)]$,
    so we let:
    \begin{equation}
    A = \frac{-\mathrm{i}m}{\hbar\Delta t}
    \begin{bmatrix}
    2 & -1 & & \\
    -1 & 2 & -1 & \\
    & \ddots & \ddots & \ddots \\
    & & -1 & 2 & -1 \\
    & & & -1 & 2
    \end{bmatrix}_{(N-1)\times(N-1)}
    \end{equation}
    and:
    \begin{equation}
    \bm{x} = \begin{bmatrix} x_{N-1} \\ \vdots \\ x_1 \end{bmatrix},
    \quad
    \bm{y} = \frac{-\mathrm{i}m}{\hbar\Delta t}
    \begin{bmatrix}
    -x_N \\ 0 \\ \vdots \\ \text{all zero} \\ \vdots \\ 0 \\ x_0
    \end{bmatrix}
    \end{equation}

    We get the new form of the exponent term:
    \begin{equation}
    \exp\left[\frac{\mathrm{i}m}{2\hbar\Delta t} \sum_{l=1}^{N-1} (x_l^2-2x_l x_{l-1} + x_{l-1}^2)\right] 
    = \exp(-\frac{1}{2}\bm{x}^T A \bm{x} - \bm{x}^T \bm{y}) e^{\frac{\mathrm{i}m}{2\hbar\Delta t}(x_N^2+x_0^2)}
    \end{equation}

    Because of $\mathcal{D}x = \prod_{l=1}^{N-1} \mathrm{d}x_l$ without $x_N$ and $x_0$, so:
    \begin{equation}
    \mathrm{i}G = \left(\frac{m}{i 2\pi\hbar\Delta t}\right)^{\frac{N}{2}} \cdot e^{\frac{\mathrm{i}m}{2\hbar\Delta t}(x_N^2+x_0^2)} \int \mathcal{D}x 
    \, e^{-\frac{1}{2}\bm{x}^T A \bm{x} - \bm{x}^T \bm{y}}
    \end{equation}

    So we can compute $(\det A) = N \cdot \left(\frac{-\mathrm{i}m}{\hbar\Delta t}\right)^N$ (Proof in \ref{appendix:special-matrix})
    and get:
    \begin{equation}
        \label{eq:path_integral_free_particles}
        \mathrm{i}G = \left(\frac{m}{i 2\pi\hbar\Delta t}\right)^{\frac{N}{2}} 
        \cdot e^{\frac{\mathrm{i}m}{2\hbar\Delta t}(x_N^2+x_0^2)} (2\pi)^{\frac{N}{2}} N^{-1/2} 
        \left(\frac{-\mathrm{i}m}{\hbar\Delta t}\right)^{-\frac{N}{2}} \cdot e^{\frac{1}{2}\bm{y}^T A^{-1} \bm{y}}
    \end{equation}

    Although $A^{-1}$ is difficult to compute, we notice that
    $\bm{y} = \begin{bmatrix} -x_N \\ 0 \\ \vdots \\ 0 \\ x_0 \end{bmatrix}$ only have two non-zero elements,
    which locate in the first row and the last row respectively.
    So we only need to calculate the first and last columns of matrix $A^{-1}$, 
    denoted $\bm{A}_1^{-1}$ and $\bm{A}_{N-1}^{-1}$, respectively (Proof in \ref{appendix:special-matrix}):
    \begin{equation}
    \bm{A}_1^{-1} = \frac{\mathrm{i}\hbar\Delta t}{mN}
    \begin{bmatrix} N-1 \\ N-2 \\ \vdots \\ 1 \end{bmatrix},
    \quad
    \bm{A}_{N-1}^{-1} = \frac{\mathrm{i}\hbar\Delta t}{mN}
    \begin{bmatrix} 1 \\ 2 \\ \vdots \\ N-1 \end{bmatrix}
    \end{equation}

    The last term of the propagator\eqref{eq:path_integral_free_particles} is:
    \begin{equation}
    \begin{aligned}
    -\frac{1}{2} \bm{y}^T A^{-1} \bm{y} &= \frac{1}{2} \bm{y}^T [x_N \bm{A}_1^{-1} + x_0 \bm{A}_{N-1}^{-1}] \\
    &= \frac{\mathrm{i}m}{2\hbar\Delta t} \left[-(x_N^2+x_0^2) + \frac{(x_N-x_0)^2}{N}\right]
    \end{aligned}
    \end{equation}

    So we can get the complete integral :
    \begin{equation}
    \begin{aligned}
    \mathrm{i}G &= \left(\frac{m}{i 2\pi\hbar\Delta t}\right)^{\frac{N}{2}} (2\pi)^{\frac{N}{2}} N^{-\frac{1}{2}} 
    \left(\frac{-\mathrm{i}m}{\hbar\Delta t}\right)^{-\frac{N-1}{2}} e^{\frac{\mathrm{i}}{\hbar} \frac{m(x_N-x_0)^2}{2N\Delta t}} \\
    &= \left(\frac{m}{i 2\pi\hbar\Delta t N}\right)^{\frac{1}{2}} e^{\frac{\mathrm{i}}{\hbar} \frac{m(x_N-x_0)^2}{2N\Delta t}}
    \end{aligned}
    \end{equation}

    where $t-t_0=N\Delta t$, $x=x_N$.
    So the free particle's propagator is:
    \begin{equation}
        \colorboxed{
        \mathrm{i}G = \left[\frac{m}{i 2\pi\hbar(t-t_0)}\right]^{\frac{1}{2}} \cdot e^{\frac{\mathrm{i}}{\hbar} \cdot \frac{m(x-x_0)^2}{2(t-t_0)}}
        }
    \end{equation}

\end{example}

\newpage
\chapter{Stationary Phase Approximation (Semiclassical Approximation)}
We have got the propagator $\mathrm{i}G$ for a partide to travel from an spacetime point $(x_0, t_0)$ to a final spacetime point $(x_f, t_f)$, 
with which is given by the Feynman path integral:
\begin{equation}
    \label{eq:path-integral-without-p}
    \colorboxed{
    \mathrm{i}G = K(x_f, t_f; x_0, t_0) 
    =\left( \frac{m}{\mathrm{i} 2\pi\hbar\Delta t} \right)^{\frac{N}{2}} \int \mathcal{D}[x(t)] \mathrm{e}^{\frac{\mathrm{i}}{\hbar} S[x(t)]}
    }
\end{equation}
where:
\begin{itemize}
    \item $x(t)$ is the passition position of the particle at time $t$, respresenting a possible path.
    \item $\mathcal{D}[x(t)]$ is the functional measure for integrating over all paths that satisfy the boundary conditions $x(t_0)=x_0$ and $x(t_f)=x_f$.
    \item $S[x(t)]$ is the action for a path $x(t)$;
\end{itemize}
This integral is infinite-dimensional and generally very difficult to calculate directly. So we need a effective method to approximate it. 
The stationary phase approximation provides a method for approximating it.

\section{One-dimensional integral of stationary phase approximation}
Consider a integral:
\begin{equation}
I = \int \mathrm{e}^{\mathrm{i} f(x)/a} \mathrm{d}x
\end{equation}
where $a$ is a small parameter and $f(x)$ is a real-valued and regular function.

Our objective is to understant the physical picture of the stationary phase approximation for the propagator path integral through this one dimensional integral stationary phase approximation.

Let's further define a new notation $\Theta(x)$ by:
\begin{equation}
\Theta(x) = \frac{1}{a} f(x)
\end{equation}
$\Theta(x)$ is a phase angle, so:
\begin{equation}
I = \int \mathrm{e}^{\mathrm{i}\Theta(x)} \mathrm{d}x
\end{equation}
This integral can be physically regarded as an interference experiment. 

Because each source at $x$ contributes a phase factor $\mathrm{e}^{\mathrm{i}\Theta(x)}$, 
so the total integral $I$ is the result of adding up all these infinite tiny vectors.

In order to compute the integral $I$, 
what we really need to do is to find the "dominant contribution" to $I$.

Firstly, let's pick up a point $x_0$ and evaluate the integral near $x_0$. 
The vicinity of $x_0$ is given by $x \in (x_0-\frac{\Delta x}{2}, x_0+\frac{\Delta x}{2})$. 
Within this small domain, we may linearize $\Theta(x)$ by a Taylor series:
\begin{equation}
\begin{aligned}
\Theta(x) &\approx \Theta(x_0) + \left.\frac{\mathrm{d}\Theta}{\mathrm{d}x}\right|_{x=x_0} (x-x_0) \\
&= \frac{f(x_0)}{a} + \frac{f'(x_0)}{a} (x-x_0)
\end{aligned}
\end{equation}

The contributions to $I$ in the small domain near $x_0$ are given by:
\begin{equation}
\begin{aligned}
I_{x_0}^{\Delta x}(x_0) &= \int_{x_0-\frac{\Delta x}{2}}^{x_0+\frac{\Delta x}{2}} \mathrm{d}x \ \mathrm{e}^{\mathrm{i}\Theta(x_0)} \cdot \mathrm{e}^{\frac{\mathrm{i}f'(x_0)}{a} (x-x_0)} \\
&= \mathrm{e}^{\mathrm{i}\Theta(x_0)} \int_{x_0-\frac{\Delta x}{2}}^{x_0+\frac{\Delta x}{2}} \mathrm{d}x \ \mathrm{e}^{\frac{\mathrm{i}f'(x_0)}{a} (x-x_0)}
\end{aligned}
\end{equation}
we let $\Theta(x_0)=\Theta_0$ and $f'(x_0)=f'_0$, so:
\begin{equation}
\begin{aligned}
I_{(x_0)}^{\Delta x} &\approx \mathrm{e}^{\mathrm{i}\Theta_0} \frac{a}{\mathrm{i}f'_0} \cdot \left. \mathrm{e}^{\frac{\mathrm{i}f'_0}{a}(x-x_0)} \right|_{x_0-\frac{\Delta x}{2}}^{x_0+\frac{\Delta x}{2}} \\
&= \mathrm{e}^{\mathrm{i}\Theta_0} \frac{2a}{\mathrm{i}f'_0} \sin \frac{f'_0 \Delta x}{2a}.
\end{aligned}
\end{equation}

Let $\alpha = \frac{f'_0 \Delta x}{2a}$, so:
\begin{equation}
I^{\Delta x}(x_0) \approx \mathrm{e}^{\mathrm{i}\Theta_0} \Delta x \cdot \frac{\sin \alpha}{\alpha}
\end{equation}

Because $a$ is a very small but nonzero parameter (in quantum mechanics, it corresponds to $\hbar$ being a very small but nonzero number),
if $f'_0=0$, $\alpha$ is strictly equal to 0 and it is not infinitely large near the zero point. 
So if we consider the case $f'(x_0)=0$, we get:
\begin{equation}
I^{\Delta x}(x_0) = \mathrm{e}^{\mathrm{i}\Theta(x_0)} \cdot \Delta x
\end{equation}
Physically, the result means that all U(1) phase in the vicinity of $x_0$ are completely the same. 
It's a perfect phase-coherence. The superposition of constant phases leads to linearly enhanced amplitude "$\Delta x$".

Next, let us consider $f'(x_0) \neq 0$. Because $a$ is a very small parameter, $\alpha = \frac{f'_0 \Delta x}{2a}$ is very large. So:
\begin{equation}
\frac{\sin\alpha}{\alpha} \rightarrow 0 \implies I^{\Delta x}(x_0) \approx 0
\end{equation}
Physically, the summation of all U(1) phases in the vicinity of $x_0$ leads to destructive interference. 
The parameter $a$ is smaller, the destructive interference is more severe.

In conclusion, if we consider small enough but nozero parameter $a$, 
it is computationlly economic to merely focus on the integral contributions from the vicinity of these special point (denoted by a set $\{x_0^i\}$) that satisfy $f'(x_0^i)=0$.

\begin{center}
\begin{tikzpicture}
\draw[->] (-0.5,0) -- (5,0) node[right] {$x$};
\draw[->] (0,-0.2) -- (0,2.5) node[above] {$I$};
\draw[domain=1:2, smooth, variable=\x] plot ({\x}, {2*exp(-20*(\x-1.5)^2)});
\node at (1.5, -0.2) {$x_0$};
\draw[domain=3:4, smooth, variable=\x] plot ({\x}, {1.8*exp(-20*(\x-3.5)^2)});
\node at (3.5, -0.2) {$x_0^i$};
\end{tikzpicture}


\end{center}

We consider the quadratic approximation in the vicinity of $x_0$:
\begin{equation}
\begin{aligned}
f(x) &\approx f(x_0) + f'(x_0)(x-x_0) + \frac{1}{2} f''(x_0)(x-x_0)^2 \\
&= f(x_0) + \frac{1}{2} f''(x_0)(x-x_0)^2
\end{aligned}
\end{equation}
As a result, the original integral $I$ can be evaluated by:
\begin{equation}
\begin{aligned}
I &\approx \sum_{\{x_0^i\}} I^{\Delta x}(x_0^i) \\
&= \sum_{\{x_0^i\}} \mathrm{e}^{\frac{\mathrm{i}}{a} f(x_0^i)} \int_{x_0^i-\frac{\Delta x}{2}}^{x_0^i+\frac{\Delta x}{2}} \mathrm{d}x \ \mathrm{e}^{\frac{\mathrm{i}}{2a} f''(x_0^i)(x-x_0^i)^2}
\end{aligned}
\end{equation}
Because parameter $a$ is a very small number, far away the stationary point $x_0^i$, 
the contributions of $\mathrm{e}^{\frac{\mathrm{i}}{2a} f''(x_0^i)(x-x_0^i)^2}$ to the integral cancel each other out through destructive interference.

So in the above Gaussian integral near $x_0$, we can extend the integral bounds to infinity:
\begin{equation}
    \colorboxed{
    \begin{aligned}
    I &\approx \sum_{\{x_0^i\}} \mathrm{e}^{\frac{\mathrm{i}f(x_0^i)}{a}} \int_{-\infty}^{\infty} \mathrm{d}x \cdot \mathrm{e}^{\frac{\mathrm{i}}{2a} f''(x_0^i)(x-x_0^i)^2} \\
    &= \sum_{\{x_0^i\}} \mathrm{e}^{\frac{\mathrm{i}f(x_0^i)}{a}} \sqrt{\frac{2\pi a \mathrm{i}}{f''(x_0^i)}}
    \end{aligned}
    }
\end{equation}

In the above integral, we use the Fresnel integral (Proof see Appendix \ref{appendix:fresnel}):
\begin{equation}
    \colorboxed{
        \int_{-\infty}^{\infty} \me^{\mi x^2} \md x = \sqrt{\frac{\mpi}{2}}(1+\mi) = \sqrt{\mpi} \me^{\mi\mpi/4}
    }
\end{equation}

\section{Semiclassical approximation of Feynman path integrals}
We have discussed that if $\hbar$ tends to zero, then the quantum system will transitions to the classical system. 
We only need to treat $f(x)$ as $S[x(t)]$ and the parameter $a$ as $\hbar$, then we can see the reason based on the discussion in the previous section. 
If $\hbar$ is zero, $I^{\Delta x}(x_0)$ equal to zero strictly for $x \neq x_0$ and is nonzero only at $x=x_0$. 
So we only need to consider the classical path with $\delta S=0$ and not need to consider the quantum fluctuation near the classical path. 
But if $\hbar$ is a very small but nonzero number, we need to consider the quantum fluctuation near the classical path. 
In other word, in classical mechanics, 
$\hbar=0$, $I^{\Delta x} \propto \frac{\sin\alpha}{\alpha}$ equal to zero in $(x_0-\frac{\Delta x}{2}, x_0) \cup (x_0, x_0+\frac{\Delta x}{2})$, 
because $\alpha = \frac{f'_0 \Delta x}{2a}$ tends to infinity no matter how small the radius of this deleted neighbourhood is. 
But in quantum mechanics, $\hbar \sim 10^{-34} \mathrm{J}\cdot\mathrm{s}$, 
$I$ is not equal to zero in the neighbourhood whose radius length matches to the order of magnitude of $\hbar$. 
Therefore, we cannot ignore the impact generated by $I$ in this neighbourhood.

The Feynman path integral:
\begin{equation}
K(x_f, t_f; x_0, t_0) \propto \int \mathcal{D}[x(t)] \cdot \mathrm{e}^{\frac{\mathrm{i}}{\hbar}S[x(t)]}
\end{equation}
it can be regarded as path integral version of $\int \mathrm{d}x \mathrm{e}^{\frac{\mathrm{i}}{a}f(x)}$. 
$f(x)$ is replaced by Classical action $S_c$ when $\delta S_c=0$. 
Liking the previous section, we consider the quadratic approximation in the vicinity of classical path:
\begin{equation}
\begin{aligned}
S &= S_c + \delta S + \frac{1}{2} \delta^2 S \\
&= S_c + \frac{1}{2} \delta^2 S.
\end{aligned}
\end{equation}

Now, we can decompose the path near classical path into the classical path $x_c(t)$ plus a quantum fluctuation $y(t)$ around it:
\begin{equation}
x(t) = x_c(t) + y(t)
\end{equation}
$x(t)$ must satisfy the same boundary conditions, so:
\begin{equation}
\begin{cases}
y(t_0) = 0 \\
y(t_f) = 0
\end{cases}
\end{equation}
let's perform a functional Taylor expansion of the action $S[x(t)]$ around the classical path $x_c(t)$:
\begin{equation}
\begin{aligned}
S[x(t)] = S[x_c+y] &= S[x_c] + \int_{t_0}^{t_f} \mathrm{d}t \left.\frac{\delta S}{\delta x(t)}\right|_{x=x_c} y(t) \\
&\quad + \frac{1}{2} \int_{t_0}^{t_f} \mathrm{d}t_1 \int_{t_0}^{t_f} \mathrm{d}t_2 \left.\frac{\delta^2 S}{\delta x(t_1) \delta x(t_2)}\right|_{x=x_c} y(t_1) y(t_2) + O(y^3)
\end{aligned}
\end{equation}
In semiclassical approximation, we assume the fluctuations $y$ are small, so we neglect terms of $O(y^3)$ and higher. At the same time, $\delta S=0$, so the action $S[x(t)]$:
\begin{equation}
S[x(t)] \approx S_c + \frac{1}{2} \int \int \mathrm{d}t_1 \mathrm{d}t_2 \ y(t_1) \cdot \left.\frac{\delta^2 S}{\delta x(t_1) \delta x(t_2)}\right|_{x=x_c} \cdot y(t_2)
\end{equation}

Now, let's solve the fluctuation term $\delta^2 S$:
\begin{equation}
\delta^2 S = \iint \mathrm{d}t_1 \mathrm{d}t_2 \cdot y(t_1) \cdot \left.\frac{\delta^2 S}{\delta x(t_1) \delta x(t_2)}\right|_{x=x_c} \cdot y(t_2)
\end{equation}
Consider a standard Lagrangian $\mathcal{L} = \frac{1}{2}m\dot{x}^2 - V(x)$, so the action is:
\begin{equation}
S = \int \mathrm{d}t \left[\frac{1}{2}m\dot{x}^2 - V(x)\right]
\end{equation}
it's second variation is:
\begin{equation}
\delta^2 S = \int \mathrm{d}t \cdot \left(\frac{1}{2}m\dot{y}^2 - \frac{1}{2}V''(x_c) \cdot y^2\right)
\end{equation}
We substitute the approximated action back into the path integral expression:
\begin{equation}
\begin{aligned}
K &\propto \int \mathcal{D}[x(t)] \exp\left\{\frac{\mathrm{i}}{\hbar}\left[S_c + \frac{1}{2}\delta^2 S\right]\right\} \\
&= \int \mathcal{D}[x(t)] \mathrm{e}^{\frac{\mathrm{i}}{\hbar}S_c} \exp\left[\int \mathrm{d}t \left(\frac{1}{2}m\dot{y}^2 - \frac{1}{2}V''(x_c)y^2\right)\right]
\end{aligned}
\end{equation}
Now, we need to change the integration variable from an integral over all paths $x(t)$ to an integral over all fluctuations $y(t)$. 
Since $x_c(t)$ is a fixed classical path, the path measure is:
\begin{equation}
\mathcal{D}[x(t)] = \mathcal{D}[x_c(t)+y(t)] = \mathcal{D}[y(t)]
\end{equation}
and $S_c$ is a constant, it can be factored out of integral:
\begin{equation}
    \colorboxed{
    \begin{aligned}
    K &\propto \mathrm{e}^{\frac{\mathrm{i}}{\hbar}S_c} \cdot \int \mathcal{D}[y(t)] \cdot \mathrm{e}^{\frac{\mathrm{i}}{2\hbar} \int \mathrm{d}t \left(\frac{1}{2}m\dot{y}^2 - \frac{1}{2}V''(x_c)y^2\right)} \\
    &= F(t_f, t_0) \cdot \mathrm{e}^{\frac{\mathrm{i}}{\hbar}S_c}.
    \end{aligned}
    }
\end{equation}
where:
\begin{enumerate}
    \item $\mathrm{e}^{\frac{\mathrm{i}}{\hbar}S_c}$ is the Classical Phase Factor. It tells us that in the semiclassical approximation, the evolution of the system's quantum phase is dominated by the classical action. This is a bridge connecting classical and quantum mechanics.
    \item $F(t_f, t_0) = \int \mathcal{D}[y(t)] \mathrm{e}^{\frac{\mathrm{i}}{2\hbar}\delta^2 S}$ is the Quantum Fluctuation Prefactor. It describes the collective contribution of all the small quantum fluctuations around the classical path.
\end{enumerate}
Now let's see a example: one dimensional free particle.
\begin{example}
    For free particle:
    \begin{equation}
    \mathcal{L}(x, \dot{x}) = \frac{1}{2}m\dot{x}^2.
    \end{equation}
    First, we need to calculate the classical action. It is given by the Euler-Lagrange equertion:
    \begin{equation}
    \frac{\mathrm{d}}{\mathrm{d}t}\left(\frac{\partial\mathcal{L}}{\partial\dot{x}}\right) - \frac{\partial\mathcal{L}}{\partial x} = 0
    \end{equation}
    For free particle:
    \begin{equation}
    \begin{cases}
    \frac{\partial\mathcal{L}}{\partial\dot{x}} = m\dot{x} \\
    \frac{\partial\mathcal{L}}{\partial x} = 0
    \end{cases}
    \end{equation}
    so the equation of motion is:
    \begin{equation}
    \frac{\mathrm{d}}{\mathrm{d}t}(m\dot{x}) = m\ddot{x} = 0
    \end{equation}
    we can get the general solution:
    \begin{equation}
    x_c(t) = at + b
    \end{equation}
    we impose the boundary conditions:
    \begin{equation}
    \begin{cases}
    x_c(t_0) = x_0 \\
    x_c(t_f) = x_f
    \end{cases}
    \end{equation}
    to determine $a$ and $b$:
    \begin{equation}
    \begin{cases}
    a = \frac{x_f - x_0}{t_f - t_0} \\
    b = x_i - t_i \left(\frac{x_f - x_0}{t_f - t_0}\right)
    \end{cases}
    \end{equation}
    the classical action is:
    \begin{equation}
    S_c = \int_{t_i}^{t_f} \mathrm{d}t \cdot \frac{1}{2}m(\dot{x}_c)^2 = \int_{t_i}^{t_f} \mathrm{d}t \cdot \frac{1}{2}ma^2 = \frac{m(x_f - x_0)^2}{2(t_f - t_0)}
    \end{equation}

    So the classical phase factor is:
    \begin{equation}
    \mathrm{e}^{\frac{\mathrm{i}}{\hbar}S_c} = \mathrm{e}^{\frac{\mathrm{i}m(x_f - x_0)^2}{2\hbar(t_f - t_i)}}
    \end{equation}
    now let's calculate the quantum fluctuation factor:
    \begin{equation}
    \begin{aligned}
    F(t_f, t_0) &= \int \mathcal{D}[y(t)] \exp\left\{\frac{\mathrm{i}m}{2\hbar} \int_{t_0}^{t_f} \left(\frac{\mathrm{d}y}{\mathrm{d}t}\right)^2 \mathrm{d}t\right\} \\
    &= \int \prod_{i=1}^{N-1} \mathrm{d}y_i \exp\left[\frac{\mathrm{i}m}{2\hbar} \sum_{i=1}^{N} \frac{(y_i - y_{i-1})^2}{\Delta t^2} \Delta t\right]
    \end{aligned}
    \end{equation}
    because $y_N = y_0 = 0$, we can rewrite the sum as:
    \begin{equation}
    \frac{\mathrm{i}m}{2\hbar} \sum_{i=1}^{N} \frac{(y_i - y_{i-1})^2}{\Delta t} = -\frac{1}{2} \bm{y}^T M \bm{y}
    \end{equation}
    where
    \begin{equation}
    M = \frac{m}{\mathrm{i}\hbar\Delta t}
    \begin{pmatrix}
    2 & -1 & & \\
    -1 & 2 & -1 & \\
    & \ddots & \ddots & \ddots \\
    & & -1 & 2 & -1 \\
    & & & -1 & 2
    \end{pmatrix}_{(N-1)\times(N-1)},
    \quad
    \bm{y} =
    \begin{pmatrix}
    y_{N-1} \\
    \vdots \\
    y_1 
    \end{pmatrix}
    \end{equation}
    and use the Gaussion integral:
    \begin{equation}
    \begin{aligned}
    F(t_f, t_0) &= (2\pi)^{(N-1)/2} (\det M)^{-1/2} \\
    &= (2\pi)^{(N-1)/2} \left(\left(\frac{\mathrm{i}\hbar\Delta t}{m}\right)^{N-1} N \right)^{-1/2}
    \end{aligned}
    \end{equation}

    So we can get the complete propagator of free partide:
    \begin{equation}
    \begin{aligned}
    K &= J \cdot \left(\frac{2\pi\mathrm{i}\hbar\Delta t}{m}\right)^{N/2} \cdot \left(\frac{m}{2\pi\mathrm{i} \hbar N\Delta t}\right)^{1/2} \cdot \mathrm{e}^{\frac{\mathrm{i}S_c}{\hbar}} \\
    &= \left(\frac{m}{2\pi\mathrm{i}\hbar\Delta t}\right)^{N/2} \cdot \left(\frac{2\pi\mathrm{i}\hbar\Delta t}{m}\right)^{N/2} \cdot \left(\frac{m}{2\pi\mathrm{i}(t_f-t_0)}\right)^{1/2} \cdot \mathrm{e}^{\frac{\mathrm{i}S_c}{\hbar}} \\
    &= \left(\frac{m}{2\pi\mathrm{i} \hbar (t_f-t_0)}\right)^{1/2} \cdot \mathrm{e}^{\frac{\mathrm{i}S_c}{\hbar}}
    \end{aligned}
    \end{equation}
    where, $S_c = \frac{m}{2} \frac{(x_f-x_0)^2}{t_f-t_0}$ 
    and $J = \left( \frac{m}{\mathrm{i} 2\pi\hbar\Delta t} \right)^{\frac{N}{2}}$
    is the coefficient in front of the path integral \ref{eq:path-integral-without-p}.

\end{example}