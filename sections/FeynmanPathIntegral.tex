\part{Feynman Path Integral}

\chapter{Feynman Path Integral}

The version of quantum mechanics:
\begin{enumerate}
    \item Schrödinger's wavefuction (operator form):
    \begin{equation}
        \hat{H}|\psi\rangle = \mathrm{i}\hbar\frac{\partial}{\partial t}|\psi\rangle
    \end{equation}
    \item Feynman's Path Integral (Common number form):
    \begin{equation}
        \mathrm{i}G(\text{Green's function}) \propto \int \mathcal{D}(x, t) \mathrm{e}^{\mathrm{i}\int \mathcal{L} \mathrm{d}t}
    \end{equation}
\end{enumerate}

There are many advantages of Feynman Path Integral:
\begin{enumerate}
    \item Make the double-slit experiment more understandable.
    \item The classical limit \enquote{$\hbar \to 0$} is \enquote{tractable}:
    quantum $\xrightarrow{\hbar \to 0}$ classical.
    \item Provide a semi-classical picture. for. quantum mechanics.
    \item \enquote{Quantum fluctuations} are more \enquote{understandable}.
    \item A natural route. to low energy effective theory of quantum many-body systems.
    \item A natural language for describing topological properties of quantum many-body systems.
\end{enumerate}

But the practical calculation in the path-integral representation
of simple quantum mechanical problem many be notoriously
difficult and lengthy.

\section{Propagators}
Consider a quantum partical confined in a one-dimensional space:
\begin{equation}
\hat{H} = \frac{\hat{p}^2}{2m} + V(\hat{x})
\end{equation}
and the canonical pair: $[\hat{x}, \hat{p}] = \hat{x}\hat{p} - \hat{p}\hat{x} = \mathrm{i}\hbar$

The Schrödinger's equation is:
\begin{equation}
\mathrm{i}\hbar \frac{\partial |\psi(t)\rangle}{\partial t} = \hat{H}|\psi(t)\rangle
\end{equation}

This first-order nature allows us to define a time evolution
operator $\hat{U}(t, t_0)$ which propagates the state vector from
an initial time $t_0$ to a final time $t$:
\begin{equation}
|\psi(t)\rangle = \hat{U}(t, t_0)|\psi(t_0)\rangle
\end{equation}

Assuming the Hamiltonian $H$ is not explicitly depend on
time, the formal solution of $\hat{U}$ is:
\begin{equation}
\hat{U} = \mathrm{e}^{-\frac{\mathrm{i}}{\hbar}\hat{H}(t-t_0)}
\end{equation}

A crucial property of $\hat{U}$ is the \enquote{chain-like} rule, or composition property. For any intermediate time $t'$ such that $t>t'>t_0$:
\begin{equation}
\hat{U}(t, t_0) = \hat{U}(t, t') \hat{U}(t', t_0)
\end{equation}

This property is the key to the entire path integral derivation.
And $\hat{U}(t, t_0)$ is unitary:
\begin{equation}
\hat{U}^{\dagger}\hat{U} = \hat{U}\hat{U}^{\dagger} = \hat{\mathbb{I}}
\end{equation}
where $\hat{\mathbb{I}}$ is the identity operator.

In the position representation, we can obtain matrix elements:
\begin{equation}
\begin{aligned}
U(x, t; x_0, t_0) &= \langle x | \hat{U}(t, t_0) | x_0 \rangle \\
&= \langle x | \mathrm{e}^{-\frac{\mathrm{i}}{\hbar}\hat{H}(t-t_0)} | x_0 \rangle
\end{aligned}
\end{equation}

We can define a propagators (Green's function) of the quantum system by using the matrix elements:
\begin{equation}
\mathrm{i}G(x, t; x_0, t_0) = U(x, t; x_0, t_0)
\end{equation}

Using the matrix elements, $\psi(x, t)$ can be reformulated as:
\begin{equation}
\begin{aligned}
\psi(x, t) &= \langle x | \hat{U}(t, t_0) | \psi(t_0) \rangle \\
&= \int \mathrm{d}x_0 \langle x | \hat{U}(t, t_0) | x_0 \rangle \langle x_0 | \psi(t_0) \rangle \\
&= \int \mathrm{d}x_0 U(x, t; x_0, t_0) \psi(x_0, t_0)
\end{aligned}
\end{equation}

Also, the propagator also satisfies the Schrödinger's equation:
\begin{equation}
    \colorboxed{\mathrm{i}\hbar \frac{\partial}{\partial t} G(x, t; x_0, t_0) = \hat{H} G(x, t; x_0, t_0)}
\end{equation}

And the initial condition is:
\begin{equation}
G(x, t_0; x_0, t_0) = -\mathrm{i} \langle x | \hat{U}(t_0, t_0) | x_0 \rangle = -\mathrm{i} \delta(x-x_0)
\end{equation}
$\delta(x-x_0)$ is Dirac function.

\begin{example}

    For free particle: $\hat{H} = \frac{1}{2m}\hat{p}^2$, in position representation:
    \begin{equation}
    \hat{p} \to -\mathrm{i}\hbar \frac{\partial}{\partial x}
    \end{equation}

    So the PDE is:
    \begin{equation}
    \mathrm{i}\hbar \frac{\partial}{\partial t} G(x, t; x_0, t_0) = -\frac{\hbar^2}{2m} \frac{\partial^2}{\partial x^2} G(x, t; x_0, t_0)
    \end{equation}

    Solve the PDE:

    Use Fourier Transform: (we use $G(x,t)$ instead of $G(x,t;x_0,t_0)$). We solve the free-particle Green's function by transforming to momentum space:
    \begin{equation}
        \left\{
        \begin{aligned}
        G(x, t) &= \frac{1}{\sqrt{2\pi}} \int \mathrm{d}k \cdot \tilde{G}(k,t) \mathrm{e}^{\mathrm{i}kx} \\
        \tilde{G}(k, t) &= \frac{1}{\sqrt{2\pi}} \int \mathrm{d}x \ G(x,t) \mathrm{e}^{-\mathrm{i}kx}
        \end{aligned}
        \right.
    \end{equation}

    With these conventions, spatial derivatives become algebraic in $k$-space while the time derivative remains unchanged:
    \begin{equation}
        \left\{
        \begin{aligned}
        \mathcal{F}\left\{ \mathrm{i}\hbar \frac{\partial G(x,t)}{\partial t} \right\} &= \mathrm{i}\hbar \frac{\partial \tilde{G}(k,t)}{\partial t} \\
        \mathcal{F}\left\{ -\frac{\hbar^2}{2m} \frac{\partial^2 G}{\partial x^2} \right\} &= -\frac{\hbar^2}{2m} [ -k^2 \tilde{G}(k,t) ] = \frac{\hbar^2 k^2}{2m} \tilde{G}(k,t)
        \end{aligned}
        \right.
    \end{equation}

    Applying the transform to the PDE yields an ordinary differential equation in time for each $k$:
    \begin{equation}
        \left\{
        \begin{aligned}
        &\mathrm{i}\hbar \frac{\partial}{\partial t} \tilde{G}(k,t) = \frac{\hbar^2 k^2}{2m} \tilde{G}(k,t)\\
        &\frac{\mathrm{d}\tilde{G}}{\tilde{G}} = -\mathrm{i} \frac{\hbar k^2}{2m} \mathrm{d}t
        \end{aligned}
        \right.
    \end{equation}

    Integrating in time gives the logarithm of the solution up to a $k$-dependent constant:
    \begin{equation}
    \ln \tilde{G} = -\mathrm{i} \frac{\hbar k^2}{2m} t + C(k)
    \end{equation}

    So we can get the solution:
    \begin{equation}
    \tilde{G}(k,t) = A(k) \mathrm{e}^{-\mathrm{i}\frac{\hbar k^2}{2m}t}, \quad A(k) = \mathrm{e}^{C(k)}
    \end{equation}

    To determine $A(k)$, impose the initial condition at time $t_0$ in position space:
    \begin{equation}
    \begin{aligned}
    \tilde{G}(k, t_0) &= \frac{1}{\sqrt{2\pi}} \int_{-\infty}^{\infty} \mathrm{d}x G(x, t_0) \mathrm{e}^{-\mathrm{i}kx} \\
    &= \frac{1}{\sqrt{2\pi}} \int_{-\infty}^{\infty} \mathrm{d}x \delta(x-x_0) \mathrm{e}^{-\mathrm{i}kx}
    \end{aligned}
    \end{equation}

    Using the Fourier transform of the Dirac delta, we find:
    \begin{equation}
    \tilde{G}(k, t_0) = \frac{1}{\sqrt{2\pi}} \mathrm{e}^{-\mathrm{i}kx_0}
    \end{equation}

    Matching at $t_0$ fixes the $k$-space amplitude:
    \begin{equation}
    A(k) = \frac{1}{\sqrt{2\pi}} \mathrm{e}^{-\mathrm{i}kx_0} \cdot \mathrm{e}^{\mathrm{i}\frac{\hbar k^2}{2m}t_0}.
    \end{equation}

    Therefore, for general time $t$ we have:
    \begin{equation}
    \tilde{G}(k,t) = \frac{1}{\sqrt{2\pi}} \mathrm{e}^{-\mathrm{i}kx_0} \mathrm{e}^{-\mathrm{i}\frac{\hbar k^2}{2m}(t-t_0)}
    \end{equation}

    Finally, inverse-transform back to position space to obtain the integral representation of the propagator:
    \begin{equation}
    \begin{aligned}
    G(x,t) &= \frac{1}{\sqrt{2\pi}} \int_{-\infty}^{\infty} \mathrm{d}k \cdot \tilde{G}(k,t) \mathrm{e}^{\mathrm{i}kx} \\
    &= \frac{1}{2\pi} \int_{-\infty}^{\infty} \mathrm{d}k \, \mathrm{e}^{\mathrm{i}k(x-x_0)} \cdot \mathrm{e}^{-\mathrm{i}\frac{\hbar(t-t_0)}{2m}k^2}
    \end{aligned}
    \end{equation}

    This is a standard Gaussian integral of the form:
    \begin{equation}
    \int_{-\infty}^{\infty} \mathrm{d}k \, \mathrm{e}^{-ak^2+bk} = \sqrt{\frac{\pi}{a}} \mathrm{e}^{\frac{b^2}{4a}}.
    \end{equation}

    Let's identify the coefficients:
    \begin{equation}
    \left\{
    \begin{aligned}
    &a = \mathrm{i} \frac{\hbar(t-t_0)}{2m} \\
    &b = \mathrm{i}(x-x_0)
    \end{aligned}
    \right.
    \end{equation}

    So we can get the solution:
    \begin{equation}
        \mathrm{i}G(x,t) 
        = \left[ \frac{m}{2\pi\hbar\mathrm{i}(t-t_0)} \right]^{\frac{1}{2}} \cdot \mathrm{e}^{\frac{\mathrm{i}}{\hbar} 
        \cdot \frac{m(x-x_0)^2}{2(t-t_0)}}
    \end{equation}

    Also, we can solve this PDE via definition:
    \begin{equation}
    \begin{aligned}
    \mathrm{i}G &= \langle x | \mathrm{e}^{-\frac{\mathrm{i}}{\hbar}(t-t_0)\hat{H}} | x_0 \rangle \\
    &= \int \langle x | \mathrm{e}^{-\frac{\mathrm{i}}{\hbar}(t-t_0)\hat{H}} | p \rangle \langle p | x_0 \rangle \, \mathrm{d}p \\
    &= \int \mathrm{d}p \, \mathrm{e}^{-\frac{\mathrm{i}(t-t_0)p^2}{2m\hbar}} \langle x | p \rangle \langle p | x_0 \rangle \\
    &= \frac{1}{2\pi\hbar} \int \mathrm{d}p \, \mathrm{e}^{-\frac{\mathrm{i}(t-t_0)p^2}{2m\hbar} + \mathrm{i}\frac{(x-x_0)}{\hbar}p}.
    \end{aligned}
    \end{equation}
    we use $P = \hbar k$ and can get the same equation as
    the Fourier Transform Method.

\end{example}


\section{Path-Integral}

When $t > t_1 > t_0$, and $t_1$ is an arbitrarily selected intermediate time,
we can write:
\begin{equation}
    \begin{aligned}
    \mathrm{i}G(x, t; x_0, t_0) &= \langle x | \hat{U}(t, t_0) | x_0 \rangle \\
    &= \langle x | \hat{U}(t, t_1) \hat{U}(t_1, t_0) | x_0 \rangle \\
    &= \int \mathrm{d}x_1 \langle x | \hat{U}(t, t_1) | x_1 \rangle \langle x_1 | \hat{U}(t_1, t_0) | x_0 \rangle \\
    &= \int \mathrm{d}x_1 \, \mathrm{i}G(x, t; x_1, t_1) \cdot \mathrm{i}G(x_1, t_1; x_0, t_0)
    \end{aligned}
\end{equation}

This intergral over $x_1$ means \enquote{superposition} of all possible \enquote{path} that connect $x$ and $x_0$.
Next, we try to \enquote{smooth} the path along time directly.
We can insert more time slices between $x$ and $x_0$.
If we insert infinite time slices, the path become smooth.

Firstly, let's discretize time. domain $[t_0, t]$ into $N$ pieces of equal length $\Delta t = \frac{t-t_0}{N}$:
\begin{equation}
\begin{aligned}
\mathrm{i}G(x, t; x_0, t_0) &= \langle x | \hat{U}(t, t_{N-1}) \hat{U}(t_{N-1}, t_{N-2}) \cdots \hat{U}(t_1, t_0) | x_0 \rangle \\
&= \int \mathrm{d}x_{N-1} \cdots \mathrm{d}x_1 \prod_{l=1}^{N} \mathrm{i}G(x_l, t_l; x_{l-1}, t_{l-1})
\end{aligned}
\end{equation}

let $\mathcal{D}_x = \prod_{l=1}^{N-1} \mathrm{d}x_l$.
Consider $N \to \infty$, so $\Delta t = \frac{t-t_0}{N} \to 0$, which means $t_l - t_{l-1} = \Delta t$.
\begin{equation}
    \label{eq:path_integral}
    \mathrm{i}G(x_l, t_l; x_{l-1}, t_{l-1}) = \langle x_l | \mathrm{e}^{-\frac{\mathrm{i}}{\hbar}\hat{H}\Delta t} | x_{l-1} \rangle
\end{equation}

Because $\Delta t$ is small, we can approximate the exponential function by its Taylor series:
\begin{equation}
    \label{eq:exp_approx}
    \mathrm{e}^{-\frac{\mathrm{i}}{\hbar}\hat{H}\Delta t} \approx \hat{\mathbb{I}} - \frac{\mathrm{i}}{\hbar}\hat{H}\Delta t 
    = \hat{\mathbb{I}} - \frac{\mathrm{i}}{\hbar}\Delta t \left[ \frac{\hat{p}^2}{2m} + V(\hat{x}) \right]
\end{equation}

Substitute \eqref{eq:exp_approx} into \eqref{eq:path_integral}:
\begin{equation}
\begin{aligned}
\mathrm{i}G(x_l, t_l; x_{l-1}, t_{l-1}) &= \int \mathrm{d}p_l \langle x_l | p_l \rangle \langle p_l | \hat{\mathbb{I}} - \frac{\mathrm{i}}{\hbar}\Delta t \left[ \frac{\hat{p}^2}{2m} + V(\hat{x}) \right] | x_{l-1} \rangle \\
&= \int \mathrm{d}p_l \langle x_l | p_l \rangle \langle p_l | x_{l-1} \rangle \left[ 1 - \frac{\mathrm{i}}{\hbar} \left( \frac{p_l^2}{2m} + V(x_{l-1}) \right) \Delta t \right] \\
\end{aligned}
\end{equation}

With the approximations $V(x_l) \approx V(x_{l-1})$:
\begin{equation}
\left[ 1 - \frac{\mathrm{i}}{\hbar} \left( \frac{p_l^2}{2m} + V(x_{l-1}) \right) \Delta t \right] 
\approx \left( 1 - \frac{\mathrm{i}}{\hbar} H_{l} \Delta t \right) \approx \mathrm{e}^{-\frac{\mathrm{i}}{\hbar}H_{l}\Delta t}
\end{equation}

So $\mathrm{i}G(x_l, t_l; x_{l-1}, t_{l-1})$ can be written as:
\begin{equation}
    \begin{aligned}
    \mathrm{i}G(x_l, t_l; x_{l-1}, t_{l-1}) &= \int \mathrm{d}p_l \frac{1}{2\pi\hbar} \mathrm{e}^{\frac{\mathrm{i}}{\hbar}p_l(x_l-x_{l-1})} \mathrm{e}^{-\frac{\mathrm{i}}{\hbar}H_{cl}\Delta t} \\
    &= \frac{1}{2\pi\hbar} \int \mathrm{d}p_l e^{\frac{\mathrm{i}}{\hbar}[p_l(x_l-x_{l-1})-H_{l}\Delta t]} \\
    &= \frac{1}{2\pi\hbar} \int \mathrm{d}p_l \mathrm{e}^{\frac{\mathrm{i}}{\hbar}[p_l(\frac{x_l-x_{l-1}}{\Delta t})-H_{l}]\Delta t}
    \end{aligned}
\end{equation}
where, $H_l$ is the classical Hamiltonion as a function of $p_l$ and $x_l$.

When $\Delta t \to 0$:
\begin{equation}
\frac{x_l - x_{l-1}}{\Delta t} = \dot{x}_l
\end{equation}

So we can get:
\begin{equation}
p_l\dot{x}_l - H_l = \mathcal{L}_l.
\end{equation}
where, $\mathcal{L}_l$ is the classical Lagrangian.

So $\mathrm{i}G(x_l, t_l; x_{l-1}, t_{l-1})$ can be written as the form with Lagrangian:
\begin{equation}
\mathrm{i}G(x_l, t_l; x_{l-1}, t_{l-1}) = \int \mathrm{d}p_l \cdot \frac{1}{2\pi\hbar} \mathrm{e}^{\frac{\mathrm{i}}{\hbar}\mathcal{L}_l\Delta t}.
\end{equation}

Substitute $\mathrm{i}G(x_l, t_l; x_{l-1}, t_{l-1})$ into the path integral:
\begin{equation}
\prod_{l=1}^{N} \mathrm{i}G(x_l, t_l; x_{l-1}, t_{l-1}) = \int \frac{\mathrm{d}p_N}{2\pi\hbar} \cdots \frac{\mathrm{d}p_1}{2\pi\hbar} \cdot \mathrm{e}^{\frac{\mathrm{i}}{\hbar}\sum_{l=1}^{N}\mathcal{L}_l \cdot \Delta t}
\end{equation}

let $\mathcal{D}_p = \prod_{l=1}^{N} \frac{\mathrm{d}p_l}{2\pi\hbar}$, when $\Delta t \to 0$, which means:
\begin{equation}
\sum_{l=1}^{N} \mathcal{L}_l \Delta t = \int_{t_0}^{t} \mathrm{d}\tau \cdot \mathcal{L}[p(\tau), x(\tau)]
\end{equation}

Finally, we can get the propagators by the path integral:
\begin{theorem}
    The propagators path integral:
    \begin{equation}
        \mathrm{i}G(x, t; x_0, t_0) 
        = \int \mathcal{D}_x \mathcal{D}_p \cdot \mathrm{e}^{\frac{\mathrm{i}}{\hbar}\int_{t_0}^{t} \mathrm{d}\tau \cdot \mathcal{L}[p(\tau), x(\tau)]}
    \end{equation}
    where, the pair of $p(t)$ and $\dot{x}(t)$ characterizes a path in the px phase space.
\end{theorem}

\section{Gaussian Integration}

If the functional integration over p is Gaussian, we can exactly integrate out p. For example, $H = \frac{p^2}{2m} + V$,
so $\mathcal{L} = p\dot{x} - H = p\dot{x} - \frac{p^2}{2m} - V(x)$,we can get:

\begin{equation}
\mathrm{i}G = \int \mathcal{D}p \mathcal{D}x \exp \left[ \frac{\mathrm{i}}{\hbar} \sum_t \left( p_t \dot{x}_t - \frac{p_t^2}{2m} - V(x_t) \right) \Delta t \right]
\end{equation}
Let:
\begin{equation}
\bm{p} = \begin{pmatrix} p_l \\ \vdots \\ p_1 \end{pmatrix}, \quad \dot{\bm{x}} = \begin{pmatrix} \dot{x}_l \\ \vdots \\ \dot{x}_1 \end{pmatrix}
\end{equation}
So we can rewrite the integral as:
\begin{equation}
\mathrm{i}G = \int \mathcal{D}x \cdot \exp \left[ \frac{\mathrm{i}}{\hbar} \sum_{l = 1}^{N} (-V(x_l)) \Delta t \right] \cdot 
\int \mathcal{D}p \cdot \exp \left[ \frac{\mathrm{i}}{2m\hbar} (-\bm{p}^T \bm{p} + 2m \bm{p}^T \dot{\bm{x}}) \Delta t \right]
\end{equation}

We have an useful formula for Gaussian integral:
\begin{equation}
\int \prod_{n = 1}^{N} \mathrm{d}x_n \exp \left[ -\frac{1}{2}\bm{x}^T A \bm{x} - \bm{x}^T \bm{y} \right] 
= (2\pi)^{\frac{N}{2}} (\det A)^{-\frac{1}{2}} \exp \left[ \frac{1}{2} \bm{y}^T A^{-1} \bm{y} \right]
\end{equation}
where, $\bm{x}, \bm{y}$ are real vectors and $A$ is real symmetric matrix.

Let:
\begin{equation}
    \begin{aligned}
    I &= \int \mathcal{D}p \exp \left[ \frac{\mathrm{i}}{2m\hbar} (-\bm{p}^T \bm{p} + 2m \bm{p}^T \dot{\bm{x}})\Delta t \right] \\
    &= \left( \frac{1}{2\pi\hbar} \right)^N \int \prod_{n = 1}^{N} \mathrm{d}p_n \cdot \exp \left[ -\frac{1}{2} \bm{p}^T A \bm{p} 
    - \bm{p}^T \dot{\bm{x}}' \right]
    \end{aligned}
\end{equation}
where $A = \frac{\mathrm{i}\Delta t}{m\hbar} \hat{\mathbb{I}}_{N \times N}$ and $ \dot{\bm{x}}' = -\frac{\mathrm{i}\Delta t}{\hbar} \dot{\bm{x}}$, 
$\hat{\mathbb{I}}_{N \times N}$ is a $N \times N$ identity matrix.

So we can get:
\begin{equation}
\left\{
\begin{aligned}
&(\det A)^{-\frac{1}{2}} = \left( \frac{\mathrm{i}\Delta t}{m\hbar} \right)^{-\frac{N}{2}}\\
&A^{-1} = \frac{m\hbar}{\mathrm{i}\Delta t} \hat{\mathbb{I}}_{N \times N}
\end{aligned}
\right.
\end{equation}

So we can get:
\begin{equation}
\begin{aligned}
\frac{1}{2} \dot{\bm{x}}'^T A^{-1} \dot{\bm{x}}' 
&= \frac{1}{2} \cdot \frac{m\hbar}{\mathrm{i}\Delta t} \cdot \left( -\frac{\mathrm{i} \Delta t}{\hbar} \right)^2 \sum_{l=1}^N \dot{x}_l^2 \\
&= \frac{\mathrm{i}}{\hbar} \sum_{l=1}^N \frac{m}{2} (\frac{x_l - x_{l-1}}{\Delta t})^2 \Delta t
\end{aligned}
\end{equation}

So:
\begin{equation}
\begin{aligned}
I &= \left( \frac{1}{2\pi\hbar} \right)^N \cdot (2\pi)^{\frac{N}{2}} \cdot \left( \frac{\mathrm{i}\Delta t}{m\hbar} \right)^{-\frac{N}{2}} 
\cdot \exp \left[ \frac{\mathrm{i}}{\hbar} \sum_{l=1}^N  \frac{m}{2} (\frac{x_l-x_{l-1}}{\Delta t})^2 \Delta t \right] \\
&= \left( \frac{m}{\mathrm{i} 2\pi\hbar\Delta t} \right)^{\frac{N}{2}} 
\exp \left[ \frac{\mathrm{i}}{\hbar} \sum_{l=1}^N \frac{m}{2} (\frac{x_l-x_{l-1}}{\Delta t})^2 \Delta t \right]
\end{aligned}
\end{equation}

So we can get the integration without $p$:
\begin{equation}
\begin{aligned}
\mathrm{i}G &= \left( \frac{m}{\mathrm{i} 2\pi\hbar\Delta t} \right)^{\frac{N}{2}} 
\int \mathcal{D}x \exp \left[ \frac{\mathrm{i}}{\hbar} \sum_{l=1}^N  \frac{m}{2} (\frac{x_l-x_{l-1}}{\Delta t})^2 \Delta t - V(x_l) \Delta t \right] \\
&= \left( \frac{m}{\mathrm{i} 2\pi\hbar\Delta t} \right)^{\frac{N}{2}} \int \mathcal{D}x 
\exp \left[ \frac{\mathrm{i}}{\hbar} \int_{t_0}^{t} \mathrm{d} \tau \, \mathcal{L}(x, \dot{x}) \right]
\end{aligned}
\end{equation}

So 
\begin{equation}
\mathrm{i}G \propto \int \mathcal{D}x \cdot e^{\frac{\mathrm{i}}{\hbar}S[x(t)]}
\end{equation}
where, $S[x(t)] = \int_{t_0}^{t} \mathrm{d}t \, \mathcal{L}(x, \dot{x})$ is action
and $\mathcal{L}(x, \dot{x}) = \frac{m\dot{x}^2}{2} - V(x)$ is Lagrangean.

From the Path-integral in real space-time, we can get some information about Physics Picture:
\begin{enumerate}[(1)]
    \item Each path is weighted with a U(1) phase factor $e^{\frac{\mathrm{i}}{\hbar}S}$.
    The Quantum interference between different paths.
    \item Since $\hbar \sim 10^{-34} \, \mathrm{J \cdot s}$, any "small change" in S (we change S to $S+\delta S$), will drastically lead to quantum destructive interference. So the only the paths that satisfy $\delta S=0$ make dominant contributions to the path-integral.
    \item Remarkably, $\delta S=0$ is exactly Hamilton's Principle in classical mechanics. So the classical paths ($\delta S=0$) dominate the path integral in the limit $\hbar \to 0$. In other words, in classical mechanics, as $\hbar \to 0$, it neglects the contribution of the integral over all other paths near the path with those with $\delta S=0$. So we can get the conclusion:
    \begin{equation*}
    \text{Quantum system} \xrightarrow{\hbar \to 0} \text{classical system}
    \end{equation*}
\end{enumerate}

\subsection{Example: Computing the path-integral of free particles}

Free particles' Hamiltonian is $H = \frac{p^2}{2m}$.
Using this Hamiltonian, we can get the path-integral:
\begin{equation}
\mathrm{i}G = \left( \frac{m}{i 2\pi\hbar\Delta t} \right)^{\frac{N}{2}} 
\int \mathcal{D}x \exp\left[\frac{\mathrm{i}}{\hbar} \sum_{l=1}^N \frac{m}{2} (\frac{x_l-x_{l-1}}{\Delta t})^2 \Delta t \right].
\end{equation}

we let:
\begin{equation}
I = \int \mathcal{D}x \exp\left[\frac{\mathrm{i}m}{2\hbar\Delta t} \sum_{l=1}^N (x_l^2 - 2x_l x_{l-1} + x_{l-1}^2)\right]
\end{equation}

In order to use Gaussian integral:
\begin{equation}
\int \prod_{n=1}^N \mathrm{d}x_n \, e^{-\frac{1}{2}\bm{x}^T A \bm{x} - \bm{x}^T \bm{y}} 
= (2\pi)^{\frac{N}{2}} (\det A)^{-\frac{1}{2}} e^{\frac{1}{2}\bm{y}^T A^{-1} \bm{y}}
\end{equation}
we should rewrite the form of $\exp[\frac{\mathrm{i}m}{2\hbar\Delta t} \sum (x_l^2-2x_l x_{l-1} + x_{l-1}^2)]$,
so we let:
\begin{equation}
A = \frac{-\mathrm{i}m}{\hbar\Delta t}
\begin{bmatrix}
2 & -1 & & \\
-1 & 2 & -1 & \\
& \ddots & \ddots & \ddots \\
& & -1 & 2 & -1 \\
& & & -1 & 2
\end{bmatrix}_{(N-1)\times(N-1)},
\quad
\bm{x} = \begin{bmatrix} x_{N-1} \\ \vdots \\ x_1 \end{bmatrix},
\quad
\bm{y} = \frac{-\mathrm{i}m}{\hbar\Delta t}
\begin{bmatrix}
-x_N \\ 0 \\ \vdots \\ \text{all zero} \\ \vdots \\ 0 \\ x_0
\end{bmatrix}
\end{equation}

We get the new form of the exponent term:
\begin{equation}
\exp\left[\frac{\mathrm{i}m}{2\hbar\Delta t} \sum_{l=1}^{N-1} (x_l^2-2x_l x_{l-1} + x_{l-1}^2)\right] 
= \exp(-\frac{1}{2}\bm{x}^T A \bm{x} - \bm{x}^T \bm{y}) e^{\frac{\mathrm{i}m}{2\hbar\Delta t}(x_N^2+x_0^2)}
\end{equation}

Because of $\mathcal{D}x = \prod_{l=1}^{N-1} \mathrm{d}x_l$ without $x_N$ and $x_0$, so:
\begin{equation}
\mathrm{i}G = \left(\frac{m}{i 2\pi\hbar\Delta t}\right)^{\frac{N}{2}} \cdot e^{\frac{\mathrm{i}m}{2\hbar\Delta t}(x_N^2+x_0^2)} \int \mathcal{D}x 
\, e^{-\frac{1}{2}\bm{x}^T A \bm{x} - \bm{x}^T \bm{y}}
\end{equation}

So we can compute $(\det A) = N \cdot \left(\frac{-\mathrm{i}m}{\hbar\Delta t}\right)^N$
and get:
\begin{equation}
    \label{eq:path_integral_free_particles}
    \mathrm{i}G = \left(\frac{m}{i 2\pi\hbar\Delta t}\right)^{\frac{N}{2}} 
    \cdot e^{\frac{\mathrm{i}m}{2\hbar\Delta t}(x_N^2+x_0^2)} (2\pi)^{\frac{N}{2}} N^{-1/2} 
    \left(\frac{-\mathrm{i}m}{\hbar\Delta t}\right)^{-\frac{N}{2}} \cdot e^{\frac{1}{2}\bm{y}^T A^{-1} \bm{y}}
\end{equation}

Although $A^{-1}$ is difficult to compute, we notice that
$\bm{y} = \begin{bmatrix} -x_N \\ 0 \\ \vdots \\ 0 \\ x_0 \end{bmatrix}$ only have two non-zero elements,
which locate in the first row and the last row respectively.

So we only need to calculate the first and last columns of matrix $A^{-1}$, denoted $\bm{A}_1^{-1}$ and $\bm{A}_{N-1}^{-1}$, respectively.
\begin{equation}
\bm{A}_1^{-1} = \frac{\mathrm{i}\hbar\Delta t}{mN}
\begin{bmatrix} N-1 \\ N-2 \\ \vdots \\ 1 \end{bmatrix},
\quad
\bm{A}_{N-1}^{-1} = \frac{\mathrm{i}\hbar\Delta t}{mN}
\begin{bmatrix} 1 \\ 2 \\ \vdots \\ N-1 \end{bmatrix}
\end{equation}

The last term of the propagator\eqref{eq:path_integral_free_particles} is:
\begin{equation}
\begin{aligned}
-\frac{1}{2} \bm{y}^T A^{-1} \bm{y} &= \frac{1}{2} \bm{y}^T [x_N \bm{A}_1^{-1} + x_0 \bm{A}_{N-1}^{-1}] \\
&= \frac{\mathrm{i}m}{2\hbar\Delta t} \left[-(x_N^2+x_0^2) + \frac{(x_N-x_0)^2}{N}\right]
\end{aligned}
\end{equation}

So we can get the complete integral :
\begin{equation}
\begin{aligned}
\mathrm{i}G &= \left(\frac{m}{i 2\pi\hbar\Delta t}\right)^{\frac{N}{2}} (2\pi)^{\frac{N}{2}} N^{-\frac{1}{2}} 
\left(\frac{-\mathrm{i}m}{\hbar\Delta t}\right)^{-\frac{N-1}{2}} e^{\frac{\mathrm{i}}{\hbar} \frac{m(x_N-x_0)^2}{2N\Delta t}} \\
&= \left(\frac{m}{i 2\pi\hbar\Delta t N}\right)^{\frac{1}{2}} e^{\frac{\mathrm{i}}{\hbar} \frac{m(x_N-x_0)^2}{2N\Delta t}}
\end{aligned}
\end{equation}

where $t-t_0=N\Delta t$, $x=x_N$.
So:
\begin{equation}
\mathrm{i}G = \left[\frac{m}{i 2\pi\hbar(t-t_0)}\right]^{\frac{1}{2}} \cdot e^{\frac{\mathrm{i}}{\hbar} \cdot \frac{m(x-x_0)^2}{2(t-t_0)}}
\end{equation}

\newpage
\chapter{Stationary Phase Approximation}