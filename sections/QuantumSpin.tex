\section{Quantum Spin}

We begin by considering the Hilbert space $\hilbert$ for a single quantum spin-1/2 particle. This is a two-dimensional complex vector space.

The conventional approach is to use an orthonormal basis formed by the eigenvectors of the spin operator along a chosen axis, typically the $z$-axis, denoted $\op{S}_z$.
\begin{equation}
    \op{S}_z \ket{\uparrow} = +\frac{\hbar}{2} \ket{\uparrow} \quad \text{and} \quad \op{S}_z \ket{\downarrow} = -\frac{\hbar}{2} \ket{\downarrow}
\end{equation}
Here, $\ket{\uparrow}$ represents the "spin up" state and $\ket{\downarrow}$ represents the "spin down" state. These two states form a complete orthonormal basis, satisfying:
\begin{itemize}
    \item \textbf{Orthogonality:} $\braket{\uparrow}{\downarrow} = 0$
    \item \textbf{Normalization:} $\braket{\uparrow}{\uparrow} = \braket{\downarrow}{\downarrow} = 1$
    \item \textbf{Completeness:} $\ketbra{\uparrow}{\uparrow} + \ketbra{\downarrow}{\downarrow} = \opI$
\end{itemize}
where $\opI$ is the identity operator in $\hilbert$.

\subsection{Spin Coherent States}

A general, normalized state $\ket{\psi}$ in the spin-1/2 Hilbert space can be written as a complex linear combination of the basis states:
\begin{equation}
    \ket{\psi} = z_1 \ket{\uparrow} + z_2 \ket{\downarrow}
\end{equation}
where $z_1, z_2 \in \mathbb{C}$ are complex coefficients.

\subsubsection{Degrees of Freedom and Normalization}

The normalization condition $\braket{\psi}{\psi} = 1$ imposes a constraint on these coefficients:
\begin{equation}
    \bra{\psi}\psi\rangle = (|z_1|^2 + |z_2|^2) = 1
\end{equation}
A complex number $z = x + \mi y$ has two real parameters. Therefore, the pair $(z_1, z_2)$ is defined by four real parameters. The normalization condition $|z_1|^2 + |z_2|^2 = 1$ removes one degree of freedom, leaving three.

Furthermore, in quantum mechanics, the overall phase of a state vector is unphysical. The states $\ket{\psi}$ and $\me^{\mi\gamma} \ket{\psi}$ (for any real $\gamma$) represent the same physical state (i.e., they belong to the same ray in Hilbert space). This "gauge freedom" removes one more degree of freedom.

This leaves $4 - 1 - 1 = 2$ real, physical degrees of freedom. This is a crucial observation: the state space of a spin-1/2 particle is topologically equivalent to the surface of a 2D sphere, which is also parameterized by two angles (like latitude and longitude).

\subsubsection{Parametrization}

We can explicitly parameterize $z_1$ and $z_2$ using two angles, $\theta$ and $\phi$, which will map directly to the surface of a sphere. A standard (but not unique) parametrization for the spin coherent state, labeled by a unit vector $\bm{n}$, is:
\begin{equation}
    \ket{\bm{n}} \equiv \ket{\theta, \phi} = \cos\left(\frac{\theta}{2}\right) \me^{-\mi\phi/2} \ket{\uparrow} + \sin\left(\frac{\theta}{2}\right) \me^{\mi\phi/2} \ket{\downarrow}
    \label{eq:coherent_state}
\end{equation}
Here, the spherical coordinate angles have the domains $\theta \in [0, \mpi]$ and $\phi \in [0, 2\mpi)$. We can easily verify that this state is normalized:
\begin{equation}
    \braket{\bm{n}}{\bm{n}} = \left|\cos\left(\frac{\theta}{2}\right) \me^{-\mi\phi/2}\right|^2 + \left|\sin\left(\frac{\theta}{2}\right) \me^{\mi\phi/2}\right|^2 = \cos^2\left(\frac{\theta}{2}\right) + \sin^2\left(\frac{\theta}{2}\right) = 1
\end{equation}
This set of states $\{\ket{\bm{n}}\}$ is continuously parameterized by the angles $(\theta, \phi)$, addressing the first drawback of the discrete basis.

\subsection{Physical Interpretation: The Bloch Sphere}

To understand the physical meaning of $\theta$ and $\phi$, we compute the expectation value of the vector spin operator $\vecS$ in the state $\ket{\bm{n}}$. We will set $\hbar = 1$ from here on for simplicity. The spin operator is $\vecS = \frac{1}{2}\vecsigma$, where $\vecsigma = (\op{\sigma}_x, \op{\sigma}_y, \op{\sigma}_z)$ is the vector of Pauli matrices:
\begin{equation}
    \op{\sigma}_x = \begin{pmatrix} 0 & 1 \\ 1 & 0 \end{pmatrix}, \quad
    \op{\sigma}_y = \begin{pmatrix} 0 & -\mi \\ \mi & 0 \end{pmatrix}, \quad
    \op{\sigma}_z = \begin{pmatrix} 1 & 0 \\ 0 & -1 \end{pmatrix}
\end{equation}
In the $\ket{\uparrow}, \ket{\downarrow}$ basis, $\ket{\bm{n}}$ is represented by the column vector:
\begin{equation}
    \ket{\bm{n}} = \begin{pmatrix} \cos(\theta/2) \, \me^{-\mi\phi/2} \\ \sin(\theta/2) \, \me^{\mi\phi/2} \end{pmatrix}
    \quad \implies \quad
    \bra{\bm{n}} = \begin{pmatrix} \cos(\theta/2) \, \me^{\mi\phi/2} & \sin(\theta/2) \, \me^{-\mi\phi/2} \end{pmatrix}
\end{equation}

\paragraph{Expectation value of $\op{S}_z$:}
\begin{equation}\begin{aligned}
    \langle \op{S}_z \rangle &= \bra{\bm{n}} \left(\frac{1}{2}\op{\sigma}_z\right) \ket{\bm{n}}
    = \frac{1}{2} \begin{pmatrix} \cos(\theta/2) \, \me^{\mi\phi/2} & \sin(\theta/2) \, \me^{-\mi\phi/2} \end{pmatrix}
      \begin{pmatrix} 1 & 0 \\ 0 & -1 \end{pmatrix}
      \begin{pmatrix} \cos(\theta/2) \, \me^{-\mi\phi/2} \\ \sin(\theta/2) \, \me^{\mi\phi/2} \end{pmatrix} \\
    &= \frac{1}{2} \left( \cos^2\left(\frac{\theta}{2}\right) - \sin^2\left(\frac{\theta}{2}\right) \right)
    = \frac{1}{2} \cos(\theta)
\end{aligned}\end{equation}

\paragraph{Expectation value of $\op{S}_x$:}
\begin{equation}\begin{aligned}
    \langle \op{S}_x \rangle &= \bra{\bm{n}} \left(\frac{1}{2}\op{\sigma}_x\right) \ket{\bm{n}}
    = \frac{1}{2} \begin{pmatrix} \dots \end{pmatrix}
      \begin{pmatrix} 0 & 1 \\ 1 & 0 \end{pmatrix}
      \begin{pmatrix} \dots \end{pmatrix} \\
    &= \frac{1}{2} \left( \cos(\theta/2)\sin(\theta/2) \me^{\mi\phi/2} \me^{\mi\phi/2} + \sin(\theta/2)\cos(\theta/2) \me^{-\mi\phi/2} \me^{-\mi\phi/2} \right) \\
    &= \frac{1}{2} \cos(\theta/2)\sin(\theta/2) \left( \me^{\mi\phi} + \me^{-\mi\phi} \right)
    = \left(\frac{1}{2} \sin\theta\right) \left(\frac{\me^{\mi\phi} + \me^{-\mi\phi}}{2}\right)
    = \frac{1}{2} \sin\theta \cos\phi
\end{aligned}\end{equation}

\paragraph{Expectation value of $\op{S}_y$:}
\begin{equation}\begin{aligned}
    \langle \op{S}_y \rangle &= \bra{\bm{n}} \left(\frac{1}{2}\op{\sigma}_y\right) \ket{\bm{n}}
    = \frac{1}{2} \begin{pmatrix} \dots \end{pmatrix}
      \begin{pmatrix} 0 & -\mi \\ \mi & 0 \end{pmatrix}
      \begin{pmatrix} \dots \end{pmatrix} \\
    &= \frac{1}{2} \left( \cos(\theta/2)(-\mi)\sin(\theta/2) \me^{\mi\phi/2} \me^{\mi\phi/2} + \sin(\theta/2)(\mi)\cos(\theta/2) \me^{-\mi\phi/2} \me^{-\mi\phi/2} \right) \\
    &= \frac{1}{2} \cos(\theta/2)\sin(\theta/2) \left( -\mi \me^{\mi\phi} + \mi \me^{-\mi\phi} \right)
    = \left(\frac{1}{2} \sin\theta\right) \left(\frac{\me^{\mi\phi} - \me^{-\mi\phi}}{2\mi}\right)
    = \frac{1}{2} \sin\theta \sin\phi
\end{aligned}\end{equation}

\subsubsection{Conclusion: The Classical Spin Vector}
Combining these results, the expectation value of the spin vector is:
\begin{equation}
    \langle \vecS \rangle = \frac{1}{2} \left( \sin\theta \cos\phi, \; \sin\theta \sin\phi, \; \cos\theta \right)
\end{equation}
This is a vector of length $S=1/2$ pointing in the direction specified by the unit vector $\bm{n}$:
\begin{equation}
    \bm{n} = (\sin\theta \cos\phi, \sin\theta \sin\phi, \cos\theta)
\end{equation}
Thus, the state $\ket{\bm{n}}$ is the quantum state that "points" in the classical direction $\bm{n}$. This direction vector lives on the surface of a unit sphere, known as the \textbf{Bloch Sphere}.

This formalism treats all directions $\bm{n}$ on an equal footing, making the SU(2) rotational symmetry manifest. This addresses the second drawback of the discrete basis.

\subsubsection{Coherent State as an Eigenvector}

The formula presented in the original notes, $\langle\bm{n}|\vecS \cdot \vecn|\bm{n}\rangle = \frac{1}{2} \ket{\bm{n}}$, appears to contain a typographical error, as an expectation value (a scalar) cannot be equal to a state vector.

The more fundamental property, which is likely intended, is the eigenvector equation:
\begin{equation}
    (\vecS \cdot \vecn) \ket{\bm{n}} = \frac{1}{2} \ket{\bm{n}}
\end{equation}
This equation signifies that the coherent state $\ket{\bm{n}}$ is, by definition, the "spin up" eigenvector of the spin operator projected along its own pointing direction $\vecn$, with the eigenvalue $+1/2$ (with $\hbar=1$).

\paragraph{Proof:}
We first construct the operator $\vecS \cdot \vecn$ in matrix form:
\begin{equation}\begin{aligned}
    \vecS \cdot \vecn &= \op{S}_x n_x + \op{S}_y n_y + \op{S}_z n_z \\
    &= \frac{1}{2} (\op{\sigma}_x n_x + \op{\sigma}_y n_y + \op{\sigma}_z n_z) \\
    &= \frac{1}{2} \left[ \begin{pmatrix} 0 & 1 \\ 1 & 0 \end{pmatrix} \sin\theta \cos\phi + \begin{pmatrix} 0 & -\mi \\ \mi & 0 \end{pmatrix} \sin\theta \sin\phi + \begin{pmatrix} 1 & 0 \\ 0 & -1 \end{pmatrix} \cos\theta \right] \\
    &= \frac{1}{2} \begin{pmatrix}
    \cos\theta & \sin\theta (\cos\phi - \mi \sin\phi) \\
    \sin\theta (\cos\phi + \mi \sin\phi) & -\cos\theta
    \end{pmatrix} \\
    &= \frac{1}{2} \begin{pmatrix}
    \cos\theta & \sin\theta \, \me^{-\mi\phi} \\
    \sin\theta \, \me^{\mi\phi} & -\cos\theta
    \end{pmatrix}
\end{aligned}\end{equation}
Now, we apply this operator to the coherent state vector $\ket{\bm{n}}$:
\begin{equation}
    (\vecS \cdot \vecn) \ket{\bm{n}} = \frac{1}{2} \begin{pmatrix}
    \cos\theta & \sin\theta \, \me^{-\mi\phi} \\
    \sin\theta \, \me^{\mi\phi} & -\cos\theta
    \end{pmatrix}
    \begin{pmatrix}
    \cos(\theta/2) \, \me^{-\mi\phi/2} \\
    \sin(\theta/2) \, \me^{\mi\phi/2}
    \end{pmatrix}
\end{equation}
We compute the top and bottom components of the resulting vector separately.

\textbf{Top component:}
\begin{equation}\begin{aligned}
    & \frac{1}{2} \left[ \cos\theta \cos(\theta/2) \me^{-\mi\phi/2} + \sin\theta \me^{-\mi\phi} \sin(\theta/2) \me^{\mi\phi/2} \right] \\
    &= \frac{1}{2} \me^{-\mi\phi/2} \left[ \cos\theta \cos(\theta/2) + \sin\theta \sin(\theta/2) \right] \\
    &= \frac{1}{2} \me^{-\mi\phi/2} \left[ \cos(\theta - \theta/2) \right] \quad \text{(using } \cos(A-B) \text{ identity)} \\
    &= \frac{1}{2} \cos(\theta/2) \me^{-\mi\phi/2}
\end{aligned}\end{equation}
This is precisely $\frac{1}{2}$ times the top component of $\ket{\bm{n}}$.

\textbf{Bottom component:}
\begin{equation}\begin{aligned}
    & \frac{1}{2} \left[ \sin\theta \me^{\mi\phi} \cos(\theta/2) \me^{-\mi\phi/2} - \cos\theta \sin(\theta/2) \me^{\mi\phi/2} \right] \\
    &= \frac{1}{2} \me^{\mi\phi/2} \left[ \sin\theta \cos(\theta/2) - \cos\theta \sin(\theta/2) \right] \\
    &= \frac{1}{2} \me^{\mi\phi/2} \left[ \sin(\theta - \theta/2) \right] \quad \text{(using } \sin(A-B) \text{ identity)} \\
    &= \frac{1}{2} \sin(\theta/2) \me^{\mi\phi/2}
\end{aligned}\end{equation}
This is precisely $\frac{1}{2}$ times the bottom component of $\ket{\bm{n}}$.

Combining both components, we have shown:
\begin{equation}
    (\vecS \cdot \vecn) \ket{\bm{n}} = \frac{1}{2} \begin{pmatrix}
    \cos(\theta/2) \me^{-\mi\phi/2} \\
    \sin(\theta/2) \me^{\mi\phi/2}
    \end{pmatrix}
    = \frac{1}{2} \ket{\bm{n}}
\end{equation}
This completes the proof. The expectation value $\langle \vecS \cdot \vecn \rangle = \bra{\bm{n}}(\vecS \cdot \vecn)\ket{\bm{n}} = \bra{\bm{n}}(\frac{1}{2}\ket{\bm{n}}) = \frac{1}{2}\braket{\bm{n}}{\bm{n}} = \frac{1}{2}$ follows directly.

\subsection{Gauge Choice and Topological Singularities}

The parametrization in Eq. \eqref{eq:coherent_state} is not unique, and it hides a subtle topological problem.
\begin{itemize}
    \item \textbf{At the North Pole} ($\theta=0$): The direction $\bm{n}$ is $(0,0,1)$. The angle $\phi$ is ill-defined. Our formula gives $\ket{\theta=0} = \cos(0)\me^{-\mi\phi/2}\ket{\uparrow} + \sin(0)\dots = \me^{-\mi\phi/2} \ket{\uparrow}$. The state vector itself depends on the meaningless angle $\phi$. This is a \textbf{singularity}.
    \item \textbf{At the South Pole} ($\theta=\mpi$): The direction $\bm{n}$ is $(0,0,-1)$. Our formula gives $\ket{\theta=\mpi} = \cos(\mpi/2)\dots + \sin(\mpi/2)\me^{\mi\phi/2}\ket{\downarrow} = \me^{\mi\phi/2} \ket{\downarrow}$. This is also singular.
\end{itemize}
This is analogous to the problem of creating a flat map of the Earth: you cannot do so without singularities (e.g., at the poles) or cuts.

We can "fix" the singularity at one pole by making a $\phi$-dependent gauge choice (i.e., multiplying by an overall phase $\me^{\mi\gamma(\phi)}$).

\paragraph{Choice 1: Regular at North Pole.}
Let's choose an overall phase $\gamma = \phi/2$. The new state, $\ket{\bm{n}}_N$, is:
\begin{equation}
    \ket{\bm{n}}_N = \me^{\mi\phi/2} \ket{\bm{n}} = \cos\left(\frac{\theta}{2}\right) \ket{\uparrow} + \sin\left(\frac{\theta}{2}\right) \me^{\mi\phi} \ket{\downarrow}
\end{equation}
\begin{itemize}
    \item At the North Pole ($\theta=0$): $\ket{\bm{n}}_N = \cos(0)\ket{\uparrow} + \sin(0)\dots = \ket{\uparrow}$. This is now regular and well-defined.
    \item At the South Pole ($\theta=\mpi$): $\ket{\bm{n}}_N = \cos(\mpi/2)\ket{\uparrow} + \sin(\mpi/2)\me^{\mi\phi}\ket{\downarrow} = \me^{\mi\phi}\ket{\downarrow}$. The singularity has been "pushed" to the South Pole.
\end{itemize}

\paragraph{Choice 2: Regular at South Pole.}
Let's choose $\gamma = -\phi/2$. The new state, $\ket{\bm{n}}_S$, is:
\begin{equation}
    \ket{\bm{n}}_S = \me^{-\mi\phi/2} \ket{\bm{n}} = \cos\left(\frac{\theta}{2}\right) \me^{-\mi\phi} \ket{\uparrow} + \sin\left(\frac{\theta}{2}\right) \ket{\downarrow}
\end{equation}
This state is regular at the South Pole ($\ket{\bm{n}}_S = \ket{\downarrow}$) but singular at the North Pole.

This unavoidable singularity is topological in nature and is the origin of the \textbf{Berry Phase}, or the "topological term," in the coherent-state path integral.

\subsection{Over-Completeness and Orthogonality}

The set of all coherent states $\{\ket{\bm{n}}\}$ for all $\bm{n}$ on the sphere is an \textbf{over-complete} basis. The Hilbert space is only 2-dimensional, but we have an infinite, continuous set of states. This means the states are not, in general, orthogonal.
\begin{equation}
    \braket{\bm{n}'}{\bm{n}} \neq 0 \quad \text{for } \bm{n}' \neq \bm{n} \text{ and } \bm{n}' \neq -\bm{n}
\end{equation}
A special exception, as noted in the text, is for antipodal states.

\subsubsection{Orthogonality of Antipodal States}
Let us prove that $\braket{-\bm{n}}{\bm{n}} = 0$. The antipodal point $-\bm{n}$ corresponds to the angles $(\theta', \phi') = (\mpi - \theta, \phi + \mpi)$.

We write the state $\ket{-\bm{n}}$ using Eq. \eqref{eq:coherent_state}:
\begin{equation}\begin{aligned}
    \ket{-\bm{n}} &= \cos\left(\frac{\mpi-\theta}{2}\right) \me^{-\mi(\phi+\mpi)/2} \ket{\uparrow} + \sin\left(\frac{\mpi-\theta}{2}\right) \me^{\mi(\phi+\mpi)/2} \ket{\downarrow} \\
    &= \cos\left(\frac{\mpi}{2}-\frac{\theta}{2}\right) \me^{-\mi\phi/2} \me^{-\mi\mpi/2} \ket{\uparrow} + \sin\left(\frac{\mpi}{2}-\frac{\theta}{2}\right) \me^{\mi\phi/2} \me^{\mi\mpi/2} \ket{\downarrow}
\end{aligned}\end{equation}
Using $\cos(\mpi/2 - x) = \sin(x)$, $\sin(\mpi/2 - x) = \cos(x)$, $\me^{-\mi\mpi/2} = -\mi$, and $\me^{\mi\mpi/2} = \mi$:
\begin{equation}
    \ket{-\bm{n}} = \sin\left(\frac{\theta}{2}\right) \me^{-\mi\phi/2} (-\mi) \ket{\uparrow} + \cos\left(\frac{\theta}{2}\right) \me^{\mi\phi/2} (\mi) \ket{\downarrow}
\end{equation}
Now we compute the inner product $\braket{-\bm{n}}{\bm{n}}$:
\begin{equation}
    \begin{aligned}
    &\braket{-\bm{n}}{\bm{n}} \\
    &=\left( \mi \sin\left(\frac{\theta}{2}\right) \me^{\mi\phi/2} 
    \bra{\uparrow} + (-\mi) \cos\left(\frac{\theta}{2}\right) \me^{-\mi\phi/2} \bra{\downarrow} \right)
    \left( \cos\left(\frac{\theta}{2}\right) \me^{-\mi\phi/2} \ket{\uparrow} + \sin\left(\frac{\theta}{2}\right) \me^{\mi\phi/2} \ket{\downarrow} \right) \\
    &= \mi \sin\left(\frac{\theta}{2}\right)\cos\left(\frac{\theta}{2}\right) \me^{\mi\phi/2}\me^{-\mi\phi/2} + (-\mi) \cos\left(\frac{\theta}{2}\right)\sin\left(\frac{\theta}{2}\right) \me^{-\mi\phi/2}\me^{\mi\phi/2} \\
    &= \mi \sin\left(\frac{\theta}{2}\right)\cos\left(\frac{\theta}{2}\right) 
    - \mi \cos\left(\frac{\theta}{2}\right)\sin\left(\frac{\theta}{2}\right) \\
    &= 0
    \end{aligned}
\end{equation}
This confirms that antipodal states are orthogonal, as expected. For example, $\ket{\bm{n}=\hat{z}} = \ket{\uparrow}$ is orthogonal to $\ket{\bm{n}=-\hat{z}} = \ket{\downarrow}$.

\subsubsection{Distinction Between \texorpdfstring{$\ket{-\bm{n}}$}{ket(-n)} and \texorpdfstring{$-\ket{\bm{n}}$}{-ket(n)}}

It is a common point of confusion to mistake the antipodal state $\ket{-\bm{n}}$ for the state $-\ket{\bm{n}}$. We must justify that, in general, $\ket{-\bm{n}} \neq -\ket{\bm{n}}$.

From our derivation in the previous section, the antipodal state is:
\begin{equation}
    \ket{-\bm{n}} = \mi \sin\left(\frac{\theta}{2}\right) \me^{-\mi\phi/2} \ket{\uparrow} - \mi \cos\left(\frac{\theta}{2}\right) \me^{\mi\phi/2} \ket{\downarrow}
\end{equation}
In contrast, the state $-\ket{\bm{n}}$ is:
\begin{equation}
    -\ket{\bm{n}} = -\cos\left(\frac{\theta}{2}\right) \me^{-\mi\phi/2} \ket{\uparrow} - \sin\left(\frac{\theta}{2}\right) \me^{\mi\phi/2} \ket{\downarrow}
\end{equation}
By simple inspection, these two state vectors are clearly not identical. They are, in fact, orthogonal to each other, as we just proved $\braket{-\bm{n}}{\bm{n}}=0$. If $\ket{-\bm{n}}$ were equal to $-\ket{\bm{n}}$, then we would have $\braket{-\bm{n}}{\bm{n}} = \braket{-\bm{n}}{-(-\bm{n})} = -1 \cdot \braket{-\bm{n}}{-\bm{n}} = -1$, which contradicts our result of $0$ (unless the state is null, which is not the case).

The state $\ket{-\bm{n}}$ represents a spin pointing in the \textit{opposite direction} (e.g., spin down), while $-\ket{\bm{n}}$ represents the \textit{same physical state} as $\ket{\bm{n}}$ but with a phase shift of $\mpi$ (since $\me^{\mi\mpi} = -1$).

\subsection{Completeness Relation}
Despite being over-complete, the spin coherent states provide a resolution of the identity operator $\opI$.

Let's call the integral $J = \int \mathrm{d}\Omega \, \ketbra{\bm{n}}{\bm{n}} = \int_0^{2\mpi} \md\phi \int_0^{\mpi} \sin\theta \, \md\theta \, \ketbra{\bm{n}}{\bm{n}}$.

\paragraph{1. Integrate over $\phi$:}
The off-diagonal terms depend on $\me^{\pm\mi\phi}$.
\begin{equation}
    \int_0^{2\mpi} \me^{\pm\mi\phi} \, \md\phi = 0
\end{equation}
The diagonal terms are independent of $\phi$.
\begin{equation}
    \int_0^{2\mpi} 1 \, \md\phi = 2\mpi
\end{equation}
After integrating over $\phi$, the matrix $J$ becomes diagonal:
\begin{equation}
    J = \int_0^{\mpi} \sin\theta \, \md\theta
    \begin{pmatrix}
    2\mpi \cos^2(\theta/2) & 0 \\
    0 & 2\mpi \sin^2(\theta/2)
    \end{pmatrix}
\end{equation}

\paragraph{2. Integrate over $\theta$:}
Both integrals evaluate to $2\mpi$. Thus, the full integral is:
\begin{equation}
    J = \int \mathrm{d}\Omega \, \ketbra{\bm{n}}{\bm{n}} = \begin{pmatrix} 2\mpi & 0 \\ 0 & 2\mpi \end{pmatrix} = 2\mpi \, \opI
\end{equation}
Dividing by $2\mpi$, we arrive at the completeness relation:
\begin{equation}
    \frac{1}{2\mpi} \int \mathrm{d}\Omega \, \ketbra{\bm{n}}{\bm{n}} = \opI
\end{equation}
This relation is the foundation for the coherent-state path integral. It allows us to insert the identity operator at infinitesimally small time steps, $t_j$, as an integral over the Bloch sphere: $\opI = \int \frac{\mathrm{d}\Omega_j}{2\mpi} \ketbra{\bm{n}_j}{\bm{n}_j}$. Summing over all paths becomes an integral over all $\bm{n}_j$ at all times $t_j$.
