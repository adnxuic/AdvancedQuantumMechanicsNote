\begin{figure}[htbp]
  \centering
  \begin{tikzpicture}[>=Stealth, scale=2]

      % --- Parameters ---
      \def\R{2.0}       % Radius of the large sphere
      \def\r{0.35}      % Radius of the monopole (inner sphere)
      \def\exAngle{15}  % Angle at which the string exits (degrees)
      \def\exRad{2.0}   % Distance to exit point (should match \R roughly)
    
      % Colors
      \definecolor{monopoleBlue}{RGB}{30, 144, 255}
    
      % Coordinates
      \coordinate (O) at (0,0);
      % Exit point on the surface of the sphere (polar coordinates)
      \coordinate (Exit) at (\exAngle:\R);
      
      % --- 1. Background Elements (Back of Sphere) ---
      % Vertical Meridian (Back half - Dashed)
      % Ellipse parameters: x-radius=0.3*R, y-radius=R
      % Drawing the left half as dashed (assuming front is right half)
      \draw[dashed, gray, line width=0.6pt] (0,\R) arc (90:270:{0.35*\R} and \R);
    
      % Horizontal Equator (Back half - Dashed)
      % Ellipse parameters: x-radius=R, y-radius=0.3*R
      % Drawing the top half as dashed
      % FIX: Added \space after \R to prevent "2.0and" concatenation error
      \draw[dashed, gray, line width=0.6pt] (\R,0) arc (0:180:\R\space and {0.35*\R});
    
    
      % --- 2. Inner Monopole (Source) ---
      % Shaded ball at the center
      \shade[ball color=monopoleBlue] (O) circle (\r);
      
      % Optional: Highlight on the monopole
      \fill[white, opacity=0.3] (0.1, 0.1) circle (0.05);
    
    
      % --- 3. Dirac String (Internal Part) ---
      % Dashed "ladder" style line from center to surface
      % We use a thick line with a specific dash pattern to mimic the "tube" look
      \draw[line width=2.5pt, dashed, dash pattern=on 3pt off 2pt] (O) -- (Exit);
      
      % Internal Arrow (Modified: changed > to < to point inward, and used \path)
      \path[decoration={markings, mark=at position 0.6 with {\arrow[scale=0.8]{<}}}, postaction={decorate}] (O) -- (Exit);
    
    
      % --- 4. Foreground Elements (Front of Sphere) ---
      % Main Sphere Outline
      \draw[thick] (O) circle (\R);
    
      % Vertical Meridian (Front half - Solid)
      % Drawing the right half
      \draw[thick] (0,-\R) arc (-90:90:{0.35*\R} and \R);
    
      % Horizontal Equator (Front half - Solid)
      % Drawing the bottom half
      % FIX: Added \space after \R to prevent "2.0and" concatenation error
      \draw[thick] (-\R,0) arc (180:360:\R\space and {0.35*\R});
    
    
      % --- 5. Exit Point Intersection ---
      % A small ring showing where the tube exits the sphere
      \begin{scope}[shift={(Exit)}, rotate=\exAngle]
        \draw[thick, fill=white] (0,0) ellipse (0.08 and 0.16);
      \end{scope}
    
    
      % --- 6. Dirac String (External Part) ---
      % Thick solid line curving outwards
      % We define a path coordinates
      \coordinate (StringEnd) at ($(Exit) + (15:2.8)$);
      
      % Draw the flux tube
      \draw[line width=3.5pt] (Exit) to[out=\exAngle, in=195] coordinate[pos=0.4] (Mid) coordinate[pos=0.85] (Tip) (StringEnd);
      
      % Draw arrows on the flux tube
      % Using markings for cleaner arrow placement on curved paths
      % Modified: Used Stealth[reversed] to point arrows inward (from outside to inside)
      \path[decoration={markings, 
          mark=at position 0.35 with {\arrow[scale=1.5, thick]{Stealth[reversed]}},
          mark=at position 0.85 with {\arrow[scale=1.5, thick]{Stealth[reversed]}}
      }, postaction={decorate}] 
      (Exit) to[out=\exAngle, in=195] (StringEnd);

      % --- 7. Labels ---
      % Label text rotated along the string
      \node[anchor=north west, rotate=12, xshift=0.2cm, yshift=-0.1cm] at (Mid) {
      \footnotesize $4\pi S$ flux tube (Dirac String)
      };

  \end{tikzpicture}
  \caption{The Dirac String}
  \label{fig:dirac_string}
\end{figure}