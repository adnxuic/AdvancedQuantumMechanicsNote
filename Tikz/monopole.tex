\begin{figure}[htbp]
    \centering
    \begin{tikzpicture}[
        >=Stealth,       % 使用 Stealth 箭头样式
        line cap=round,  % 线条端点圆滑
        thick            % 线条加粗
    ]

        % --- 定义参数 ---
        \def\R{2.5}      % 球体半径
        \def\YScale{0.3} % 赤道椭圆的压扁程度 (用于模拟3D透视)

        % --- 1. 绘制球体轮廓 ---
        % 主圆
        \draw[black] (0,0) circle (\R);

        % --- 2. 绘制赤道 (3D 效果) ---
        % 后半部分 (虚线,表示被球体遮挡)
        \draw[dashed, black] (\R,0) arc (0:180:\R cm and \R*\YScale cm);
        % 前半部分 (实线)
        \draw[black] (\R,0) arc (0:-180:\R cm and \R*\YScale cm);

        % --- 3. 绘制单极子 (中心点) ---
        % 画一个红色的实心圆点
        \fill[red] (0,0) circle (0.15);

        % --- 4. 绘制磁场/电场线 (辐射状箭头) ---
        % 使用红色箭头,长度和角度略有不同以模拟立体感
        \begin{scope}[red, ->, line width=1.2pt]
            % 垂直方向
            \draw (0,0) -- (0, 1.6);
            \draw (0,0) -- (0, -1.6);
            
            % 水平方向 (略带透视角度)
            \draw (0,0) -- (1.7, 0.4);
            \draw (0,0) -- (-1.7, 0.4);
            
            % 对角线方向
            \draw (0,0) -- (1.2, 1.3);
            \draw (0,0) -- (-1.2, 1.3);
            \draw (0,0) -- (1.3, -1.0);
            \draw (0,0) -- (-1.3, -1.0);
            
            % 指向读者的短箭头 (透视缩短)
            \draw (0,0) -- (0.4, -0.5);
        \end{scope}

        % --- 5. 添加标注 "monopole" ---
        % 节点位置在右上方
        \node[red, anchor=south west, font=\Large\itshape] (label) at (2, 2) {monopole};
        
        % 修改点: 从单极子中心指向文本
        % out=45, in=190 控制箭头的弯曲角度,使其从中心以 45 度方向出发,到达文本时以 190 度角度进入
        \draw[->, red, thick] (0.2, 0.2) to[out=45, in=190] (label.west);

    \end{tikzpicture}
    \caption{The monopole}
    \label{fig:monopole}
\end{figure}