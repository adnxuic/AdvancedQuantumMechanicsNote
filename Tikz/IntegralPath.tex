% 设置视角参数
\tdplotsetmaincoords{70}{110}

\begin{figure}[htbp]
    \centering
    \begin{tikzpicture}[
        background rectangle/.style={fill=boardbg}, show background rectangle,
        scale=1,
        line cap=round,
        line join=round,
        % 箭头样式
        midarrow/.style={
            decoration={
                markings,
                mark=at position 0.6 with {\arrow[scale=1.2, >=stealth]{>}}
            },
            postaction={decorate}
        }
    ]
    
        % --- 1. 坐标定义 (保持不变) ---
        \def\R{2.5} 
        \coordinate (O) at (0,0);
        \coordinate (N) at (0, \R);   
        \coordinate (S) at (0, -\R);  
        \coordinate (ni) at (1.1, 0.9);   
        \coordinate (nf) at (-0.9, -0.4); 
        \coordinate (midpath) at (0.1, 0.1); 
    
        % --- 2. 绘制球体基础 ---
        
        % 球体轮廓 (黑色)
        \draw[sphereoutline, thick] (O) circle (\R);
        
        % 背景纬线 (颜色加深)
        \foreach \y in {0.5, 1.0, 1.5, 2.0} {
            \draw[latpink, thin] ({-sqrt(\R*\R-\y*\y)}, \y) arc (180:360: {sqrt(\R*\R-\y*\y)} and 0.15);
        }
        \foreach \y in {-0.5, -1.0, -1.5, -2.0} {
            \draw[latpink, thin] ({-sqrt(\R*\R-\y*\y)}, \y) arc (180:360: {sqrt(\R*\R-\y*\y)} and 0.15);
        }
        
        % 赤道 (深金色)
        \draw[equator, thick] (-\R,0) arc (180:360: \R cm and 0.35cm);
        \draw[equator, thin, dashed] (\R,0) arc (0:180: \R cm and 0.35cm); % 背面用虚线
    
        % --- 3. 绿色区域绘制 ---
        
        % 定义封闭路径
        \def\boundarypath{
            (N) 
            to[out=-65, in=105] (ni)              
            to[out=-110, in=60] (midpath)         
            to[out=-120, in=25] (nf)              
            to[out=80, in=-115] (N)               
            -- cycle
        }
    
        % A. 填充 (淡绿色)
        \fill[areafill, opacity=0.6] \boundarypath;
    
        % B. 内部网格 (剪裁区域)
        \begin{scope}
            \clip \boundarypath;
            
            % 经线 (深绿色细线)
            \draw[gridgreen, thin] (N) to[out=-75, in=80] (0.5, -1);
            \draw[gridgreen, thin] (N) to[out=-85, in=90] (0.0, -1);
            \draw[gridgreen, thin] (N) to[out=-95, in=100] (-0.4, -1);
            \draw[gridgreen, thin] (N) to[out=-105, in=110] (-0.8, -1);
            
            % 纬线
            \foreach \h in {2.0, 1.6, 1.2, 0.8, 0.4, 0.0, -0.4} {
                 \draw[gridgreen, thin] (-2, \h) arc (180:360: 2 and 0.25);
            }
        \end{scope}
    
        % C. 区域边界线 (深绿色粗线)
        \draw[boundarygreen, thick] (N) to[out=-65, in=105] (ni); 
        \draw[boundarygreen, thick] (N) to[out=-115, in=80] (nf); 
    
        % --- 4. 路径 gamma ---
        \draw[pathblue, very thick, midarrow] (ni) to[out=-110, in=60] (midpath);
        \draw[pathblue, very thick, midarrow] (midpath) to[out=-120, in=25] (nf);
    
        % --- 5. 点与标注 (颜色加深以适配白底) ---
        
        % S & N 点
        \node[circle, fill=sphereoutline, inner sep=1.5pt] at (S) {};
        \node[sphereoutline, above] at (S) {$S$};
    
        \node[circle, fill=sphereoutline, inner sep=1.5pt] at (N) {};
        \node[sphereoutline, above=1pt] at (N) {$N$};
    
        % nf 点
        \node[circle, fill=pointyellow, draw=black, thin, inner sep=2.5pt] (Pnf) at (nf) {};
        % nf 标签 (金色/深黄引线)
        \draw[<-, equator, thick] (Pnf) to[out=-150, in=10] (-2.8, -1.8) node[left, pointyellow, font=\huge] {$n_f$};
    
        % ni 点
        \node[circle, fill=areafill, draw=boundarygreen, thin, inner sep=2pt] (Pni) at (ni) {};
        % ni 标签 (深绿引线)
        \draw[<-, boundarygreen, thick] (Pni) -- ++(0.8, 0.6) node[right, textgreen, font=\Large] {$n_i$};
    
        % Area 标签
        \node[textgreen, font=\Huge] (LblArea) at (-2.5, 2.8) {$A_\gamma$};
        \node[textgreen, font=\large] at (-2.5, 2.3) {area};
        \draw[->, textgreen, thick, bend left=20] (LblArea.east) to (-0.4, 1.8);
    
        % Path 标签
        \node[pathblue, font=\Huge, right] (LblPath) at (2.2, 2.8) {path $\gamma$};
        \draw[->, pathblue, thick, bend left=15] (LblPath.west) to (0.8, 0.5);
    
    \end{tikzpicture}
    \caption{The integral path $\gamma$ in the Bloch sphere}
    \label{fig:IntegralPathInBlochSphere}
\end{figure}