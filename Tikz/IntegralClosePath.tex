\begin{figure}[htbp]
    \centering
    \begin{tikzpicture}[
        scale=1.8,
        line cap=round,
        line join=round,
        % 定义路径上的箭头样式 (多个箭头表示流向)
        flowarrow/.style={
            decoration={
                markings,
                mark=at position 0.15 with {\arrow[scale=1, >=stealth]{>}},
                mark=at position 0.45 with {\arrow[scale=1, >=stealth]{>}},
                mark=at position 0.75 with {\arrow[scale=1, >=stealth]{>}}
            },
            postaction={decorate}
        }
    ]
    
        % --- 1. 基础几何定义 ---
        \def\R{2.5} 
        \coordinate (O) at (0,0);
        \coordinate (N) at (0, \R);   
        \coordinate (S) at (0, -\R);  
    
        % --- 2. 绘制球体 ---
        
        % 球体轮廓
        \draw[sphereoutline, thick] (O) circle (\R);
        
        % 纬线 (模拟手绘的透视感,上半球多画几条)
        \foreach \y in {0.6, 1.2, 1.7, 2.1} {
            \draw[latpink, thin] ({-sqrt(\R*\R-\y*\y)}, \y) arc (180:360: {sqrt(\R*\R-\y*\y)} and 0.15);
        }
        \foreach \y in {-0.6, -1.2, -1.8} {
            \draw[latpink, thin] ({-sqrt(\R*\R-\y*\y)}, \y) arc (180:360: {sqrt(\R*\R-\y*\y)} and 0.15);
        }
        
        % 赤道
        \draw[equator, thick] (-\R,0) arc (180:360: \R cm and 0.35cm);
        \draw[equator, thin, dashed, opacity=0.6] (\R,0) arc (0:180: \R cm and 0.35cm);
    
        % --- 3. 绘制闭合路径与区域 ---
        
        % 定义闭合形状的控制点
        % 这是一个位于上半球的不规则闭合曲线
        \def\closedloop{
            plot [smooth cycle, tension=0.8] coordinates {
                (-0.8, 1.0)  % 左侧起点 (n_i)
                (-0.3, 1.5)  % 上方凹陷处
                (0.4, 1.7)   % 右上
                (1.1, 1.2)   % 右侧
                (0.5, 0.7)   % 右下
                (-0.4, 0.6)  % 左下
            }
        }
    
        % A. 区域填充 (使用阴影线 pattern)
        % pattern color 控制线条颜色
        \fill[pattern=north west lines, pattern color=textgreen!60] \closedloop;
        % 也可以加一层淡蓝色背景增加可读性
        \fill[textgreen, opacity=0.1] \closedloop;
    
        % B. 绘制路径线条 (蓝色粗线,带箭头)
        \draw[pathblue, very thick, flowarrow] \closedloop;
    
        % --- 4. 极点 (改为点) ---
        
        % 北极点
        \node[circle, fill=sphereoutline, inner sep=1.5pt] at (N) {};
        \node[above=2pt] at (N) {$N$};
        
        % 南极点
        \node[circle, fill=sphereoutline, inner sep=1.5pt] at (S) {};
        \node[below=2pt] at (S) {$S$};
    
        % --- 5. 其他标注 ---
    
        % 标记点 n_f = n_i
        \coordinate (Ni) at (-0.9, 1.02); % 对应路径上的点
        \node[circle, fill=pathblue, inner sep=2.5pt] (PointNi) at (Ni) {};
        
        % 标签: n_f = n_i
        \draw[<-, pathblue, thick] (PointNi) -- ++(-0.8, 0.1) node[left, pathblue, font=\Large] {$n_f = n_i$};
    
        % 标签: Ar
        \node[labelgreen, font=\Huge] (LblAr) at (2.2, 2.2) {$A_\gamma$};
        % 指向区域中心
        \draw[->, labelgreen, thick, bend right=20] (LblAr.west) to (0.3, 1.2);
    
    
    \end{tikzpicture}
    \caption{The integral close path $\gamma$ in the Bloch sphere}
    \label{fig:IntegralClosePathInBlochSphere}
\end{figure}