\begin{figure}[htbp]
    \centering
    \begin{tikzpicture}[
        font=\sffamily,
        >=Stealth,
        % 样式定义
        % 1. 球体样式升级:添加 ball color 产生3D光影,opacity 控制透明度以便看到背后的网格(如果需要)
        sphere body/.style={shading=ball, ball color=yellow!10, opacity=0.4},
        sphere outline/.style={color=yellow!90!black, thick},
        latitude/.style={color=red!60, thin},
        equator/.style={color=yellow!90!black, thick},
        region outline/.style={color=blue, very thick},
        region fill/.style={pattern=north east lines, pattern color=blue!40},
        arrow style/.style={color=blue, thick, ->}
    ]

        % 定义球体半径
        \def\R{3}
        
        % 0. (可选) 底部阴影,增加落地感
        \fill[black!10] (0, -\R-0.2) ellipse ({0.8*\R} and 0.1*\R);

        % 1. 绘制球体本体 (3D 填充)
        % 这步是立体感的关键
        \shade[sphere body] (0,0) circle (\R);
        \draw[sphere outline] (0,0) circle (\R);

        % 2. 绘制纬线 (Latitudes) - 升级为椭圆弧
        % 视角设定:假设我们稍微从上方俯视,纬线会呈现向下弯曲的弧度
        % y radius = 0.15 * x radius 模拟透视压缩
        \foreach \h in {-2.4, -1.8, ..., 2.4} {
            % 计算当前高度对应的水平半径 w
            \pgfmathsetmacro{\w}{sqrt(\R*\R - \h*\h)}
            
            % 使用椭圆弧替代 bend right
            % (180:360:...) 表示下半圆弧,产生向下弯曲的透视效果
            \draw[latitude] ({-\w}, \h) arc (180:360:{\w} and {0.15*\w});
        }

        % 3. 绘制赤道 (Equator)
        % 同样使用椭圆弧,加粗显示
        \draw[equator] ({-\R}, 0) arc (180:360:{\R} and {0.15*\R});

        % 标注极点 N 和 S
        \node[yellow!80!black, above] at (0, \R) {N};
        \node[yellow!80!black, below] at (0, -\R) {S};

        % 4. 绘制不规则区域 A_gamma (蓝色区域)
        \begin{scope}
            % 微调坐标以适应新的透视感
            \coordinate (A) at (-1.2, 0.2);  
            \coordinate (B) at (0.0, -0.3); 
            \coordinate (C) at (1.5, 0.8);  
            \coordinate (D) at (0.8, 1.8);  
            \coordinate (E) at (-0.8, 1.5); 

            % 定义路径
            \def\regionpath{plot[smooth cycle, tension=0.7] coordinates {(A) (B) (C) (D) (E)}}

            % 填充阴影
            \fill[region fill] \regionpath;

            % 绘制带箭头的轮廓
            \draw[region outline, decoration={
                markings,
                mark=at position 0.2 with {\arrow{>}},
                mark=at position 0.55 with {\arrow{>}},
                mark=at position 0.85 with {\arrow{>}}
            }, postaction={decorate}] \regionpath;

            % 5. 绘制点 n_f = n_i
            \node[circle, fill=blue, inner sep=2pt] (point) at (A) {};
        \end{scope}

        % 6. 添加外部标注和指示箭头
        \node[right, text=green!60!black, font=\Large] (LabelAg) at (\R+0.5, \R-0.5) {$A_\gamma$};
        \draw[arrow style] (LabelAg.west) to[bend right=20] (0.5, 1.0);

        \node[left, text=blue, font=\large] (LabelN) at (-\R-0.5, 0.2) {$n_f = n_i$};
        \draw[arrow style] (LabelN.east) to[bend left=10] (point);

    \end{tikzpicture}
    \caption{The integral close path $\gamma$ in the Bloch sphere}
    \label{fig:IntegralClosePathInBlochSphere}
\end{figure}