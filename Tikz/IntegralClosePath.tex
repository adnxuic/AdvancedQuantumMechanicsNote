\begin{figure}[htbp]
    \centering
    \begin{tikzpicture}[
        scale=1,
        line cap=round,
        line join=round,
        % 箭头样式
        flowarrow/.style={
            decoration={
                markings,
                mark=at position 0.14 with {\arrow[scale=1, >=stealth]{>}},
                mark=at position 0.45 with {\arrow[scale=1, >=stealth]{>}},
                mark=at position 0.78 with {\arrow[scale=1, >=stealth]{>}}
            },
            postaction={decorate}
        }
    ]
    
        % --- 1. 基础参数 ---
        \def\R{2.5} 
        \coordinate (O) at (0,0);
        \coordinate (N) at (0, \R);   
        \coordinate (S) at (0, -\R);  
    
        % --- 2. 绘制球体背景 ---
        % 轮廓
        \draw[sphereoutline, thick] (O) circle (\R);
        
        % 纬线
        \foreach \y in {0.6, 1.2, 1.7, 2.1} {
            \draw[latpink, thin] ({-sqrt(\R*\R-\y*\y)}, \y) arc (180:360: {sqrt(\R*\R-\y*\y)} and 0.15);
        }
        \foreach \y in {-0.6, -1.2, -1.8} {
            \draw[latpink, thin] ({-sqrt(\R*\R-\y*\y)}, \y) arc (180:360: {sqrt(\R*\R-\y*\y)} and 0.15);
        }
        
        % 赤道
        \draw[equator, thick] (-\R,0) arc (180:360: \R cm and 0.35cm);
        \draw[equator, thin, dashed, opacity=0.6] (\R,0) arc (0:180: \R cm and 0.35cm);
    
        % --- 3. 闭合区域 Ar ---
        
        % 【关键修改】预先定义起点坐标,确保点和线重合
        \coordinate (StartPoint) at (-0.95, 1.0);
    
        % 定义路径 (包含起点 StartPoint)
        \def\closedloop{
            plot [smooth cycle, tension=0.7] coordinates {
                (StartPoint)   % 使用定义好的坐标
                (-0.3, 1.55) 
                (0.5, 1.65)  
                (1.2, 1.1)   
                (0.6, 0.6)   
                (-0.3, 0.65) 
            }
        }
    
        % A. 区域填充 (淡蓝底色 + 阴影线)
        \fill[textgreen!60, opacity=0.1] \closedloop;
        \fill[pattern=north west lines, pattern color=textgreen!60] \closedloop;
    
        % B. 路径轮廓
        \draw[pathblue, very thick, flowarrow] \closedloop;
    
        % --- 4. 极点 (实心圆点) ---
        \node[circle, fill=sphereoutline, inner sep=1.5pt] at (N) {};
        \node[above=2pt] at (N) {$N$};
        
        \node[circle, fill=sphereoutline, inner sep=1.5pt] at (S) {};
        \node[below=2pt] at (S) {$S$};
    
        % --- 5. 标注 ---
    
        % 标记点 n_f = n_i (直接放置在 StartPoint 上)
        \node[circle, fill=pathblue, inner sep=2.5pt] (PointNi) at (StartPoint) {};
        
        % 标签: n_f = n_i
        \draw[<-, pathblue, thick] (PointNi) -- ++(-0.8, 0.1) node[left, pathblue, font=\Large] {$n_f = n_i$};
    
        % 标签: Ar
        \node[labelgreen, font=\Huge] (LblAr) at (2.2, 2.2) {$A_\gamma$};
        \draw[->, labelgreen, thick, bend right=20] (LblAr.west) to (0.3, 1.2);
    
    \end{tikzpicture}
    \caption{The integral close path $\gamma$ in the Bloch sphere}
    \label{fig:IntegralClosePathInBlochSphere}
\end{figure}