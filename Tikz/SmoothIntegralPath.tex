\begin{figure}[htbp]
    \centering
    \begin{tikzpicture}[
        scale=1.5,
        >=Stealth,
        % Style for placing an arrow in the middle of a path
        midarrow/.style={
            decoration={
                markings,
                mark=at position 0.6 with {\arrow[scale=1.2]{>}}
            },
            postaction={decorate}
        },
        axis/.style={->, thick, draw=black!80},
        label node/.style={font=\small\itshape}
    ]

        % --- COORDINATES ---
        \coordinate (Origin) at (0,0); % (tc, xc)
        \coordinate (Target) at (3,4); % Destination point

        % --- AXES ---
        % Vertical Axis (Time)
        \draw[axis] (0, -0.5) -- (0, 5.5) node[left] {\Large Time};
        % Horizontal Axis (Space)
        \draw[axis] (-0.5, 0) -- (5, 0) node[below] {\Large Space};

        % --- TICKS AND LABELS ---
        % Origin labels (tc, xc)
        \node[below left] at (Origin) {\large $t_c, x_c$};
        \filldraw[black] (Origin) circle (1.5pt);

        % Target labels (ta, xa) projections
        \draw[dashed, gray] (Target) -- (0, 4) node[left, black] {\large $t_a$};
        \draw[dashed, gray] (Target) -- (3, 0) node[below, black] {\large $x_a$};
        \draw[dashed, gray] (0,0) -- (0,0); % Dummy for anchor

        % --- PATHS ---

        % 1. Quantum Paths (The "Sum over histories")
        % Using Bezier curves to simulate random paths
        
        % Path Left 1
        \draw[thick, midarrow, gray!80] (Origin) .. controls (-1, 1) and (-0.5, 3) .. (Target);
        
        % Path Left 2 (Wider)
        \draw[thick, midarrow, gray!80] (Origin) .. controls (-2, 1.5) and (-1, 3.5) .. (Target);
        
        % Path Right 1
        \draw[thick, midarrow, gray!80] (Origin) .. controls (1.5, 0.5) and (2.5, 2) .. (Target);
        
        % Path Right 2 (Wider loop)
        \draw[thick, midarrow, gray!80] (Origin) .. controls (4, 1) and (4.5, 3) .. (Target);
        
        % Path Middle Wiggle
        \draw[thick, midarrow, gray!80] (Origin) .. controls (0.5, 2) and (2.5, 2) .. (Target);

        % Path Cross-over
        \draw[thick, midarrow, gray!80] (Origin) .. controls (1, 3) and (-1, 2) .. (Target);

        % 2. Classical Path (The Red Line - Newton's Law)
        % Drawn last to be on top
        \draw[red, very thick, midarrow] (Origin) -- (Target);
        
        % Target Point Dot
        \filldraw[white, draw=black, thick] (Target) circle (2pt);

        % --- ANNOTATION TEXT ---
        % "This red path is 'classical path'..."
        \node[align=left, text=red!80!black, font=\small] (annotation) at (5.5, 4.5) {
            This red path is ``classical path'' \\
            for free particle \\
            (Newton's law)
        };

        % Arrow from annotation to the red path
        \draw[->, red!80!black, dashed, thick] (annotation.west) to[bend right=20] (1.6, 2.2);

    \end{tikzpicture}
    \caption{Schematic of Feynman Path Integral. The red path is the classical path (Newton's law) for a free particle, while the gray paths represent the quantum sum over histories.}
    \label{fig:path_integral}
\end{figure}